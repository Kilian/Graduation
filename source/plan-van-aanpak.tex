\documentclass[a4paper, 10pt, twoside, pdftex]{article}
	\usepackage[dutch]{babel}
	\usepackage{fullpage}
	\linespread{1.04}

\title{\textsc{Gebruiksvriendelijkheid van Sociale Netwerken} \linebreak $Plan~van~Aanpak$}
\author{Kilian Valkhof}
\date{17 augustus 2009}
\makeindex

\begin{document}

\maketitle

\begin{quote}
Good Judgment comes from Experience. Experience comes from Bad Judgment. 
\end{quote}
\newpage 
\tableofcontents

\newpage
\section{Aanleiding}
In de recente jaren zijn er een tweetal dingen gebeurd: sociale netwerken hebben een explosieve groei doorgemaakt 
en gebruiksvriendelijkheid voor websites hebben terecht een veel grotere nadruk gekregen dan daarvoor. Hoewel mensen 
als Jacob Neilsen en Jesse James Garret het ons al jaren vertellen, is het pas de afgelopen jaren `normaal' geworden
om ook aandacht te besteden aan gebruiksvriendelijkheid. Veel van de theorie op dit gebied focust zich echter op 
informatieve websites (nieuwssites, bedrijfswebsites en weblogs) en minder op de nieuwe vorm van websites: social networks.

Doordat mensen op sociale netwerken heel anders bezig zijn dan op gewone informatieve websites --- in de laatste wordt er 
passief gelezen, in de eerste vorm is er sprake van interactie --- leek het mij dat dit gevolgen heeft voor de gebruiksvriendelijkheid
van sociale netwerken. Mijn afstudeerbedrijf, $Wakoopa$, is zo'n sociaal netwerk, en is daarom met mij geinteresseerd in het verbeteren
van de gebruikersvriendelijkheid van hun site en het uitvinden welke regels er gelden rondom de gebruikersvriendelijkheid van sociale netwerken.

%TODO relatie leggen met minor en competenties

\section{Doelstelling}
Het doel van deze afstudeerstage is het uitvinden wat gebruiksvriendelijkheid voor sociale netwerken anders maakt, en welke regels voor dit type websites gelden. 
Ik hoop met een set aanbevelingen te kunnen komen die specifiek gericht zijn sociale netwerken, en deze aanbevelingen op de site van Wakoopa door te kunnen voeren als casus.


\section{Probleemstelling}
Deze afstudeerstage heeft de volgende onderzoeksvraag:
\begin{quote}
 \textbf{Hebben sociale netwerken andere eisen op het gebied van gebruiksvriendelijkheid dan de huidige inzichten op webgebied, welke zijn dat en hoe verhoud het sociaal netwerk Wakoopa zich tot deze eisen?}
\end{quote}


\section{Onderzoeksvragen}
\begin{quote}
 \textbf{Wat zeggen andere onderzoeken op het gebied van usability en sociale netwerken?}
\end{quote}
Er zijn een aantal onderzoeken over sociale netwerken in het algemeen, en een aantal usability onderzoeken specifiek voor Wakoopa. Ik begin met een analyse van deze onderzoeken, waar mogelijk aan de hand van de Heuristics van Nielssen, hier komen een aantal globaal geldende conclusies en een aantal specifiek voor Wakoopa.
    
\begin{quote}
\textbf{Wat vinden gebruikers van het sociaal netwerk Wakoopa op het gebied van usability?}
\end{quote}
Door gebruikersonderzoek binnen de leden van Wakoopa doe ik onderzoek naar hun meningen. Dit wil ik kwantitatief d.m.v. A/B-testen, statistiekanalyse en een survey en analyse daarvan, en kwalitatief door middel van interviews en remote/lab tests onderzoeken.

    
\begin{quote}
\textbf{Hoe verhouden de bevindingen van de onderzoeken zich tot algemeen geaccepteerde usability standaarden?}
\end{quote}
Hierin kijk ik naar welke usabilitystandaarden voor het wel web er nu zijn. Na dit ge\"{i}ntroduceerd te hebben zal ik de conclusies uit de vorige hoofdstukken vergelijken met deze standaarden.

    
\begin{quote}
\textbf{Welke verbeteringen zijn er specifiek voor Wakoopa door te voeren?}
\end{quote}
Kijkend naar de conclusies van de vorige hoofdstukken, hoe kan Wakoopa verbeterd worden?

    
\begin{quote}
\textbf{Welke eisen hebben sociale netwerken op het gebied van usability?}
\end{quote}
Hierin beantwoord ik het belangrijkste gedeelte van de hoofdvraag, met conclusies gebaseerd op de vorige hoofdstukken


\begin{quote}
\textbf{Wat zijn de quick wins op het gebied van usability voor sociale netwerken?}
\end{quote}
In dit hoofdstuk wil ik een checklist voor sociale netwerken opzetten waarmee andere bedrijven hun sociale netwerk onder de loep kunnen nemen.

\section{Betrokken partijen}
De betrokken partijen bij deze afstudeerscriptie zijn:
\begin{center}
\begin{tabular}{rlll}
× & Functie & E-mail & Telefoonnummer\\
Kilian Valkhof & Afstudeerder & kilian@wakoopa.com & 06 40 96 84 20\\
Robert Gaal & Bedrijfsbegeleider & Robert@wakoopa.com & 06 41 24 23 58\\
Sandra Hekkelman & Stagebegeleider & × & ×
\end{tabular}
\end{center}

\section{Werkwijze en Planning}
Deze afstudeerstage bestaat uit een aantal fasen:

\begin{itemize}
\item 
Analyse van bestaande literatuur
\item 
Onderzoek onder gebruikers van Wakoopa
\item 
Analyseren onderzoeksresultaten
\item 
Vergelijken met bestaande literatuur
\item 
Doorvoeren op de website van Wakoopa
\item 
Evalueren
\end{itemize}


\section{Evaluatie}



\end{document}
