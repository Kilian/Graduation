\documentclass[a4paper, 10pt, pdftex]{article}
  \usepackage[dutch]{babel}
  %\usepackage{fullpage}
  \usepackage{ulem}
  \usepackage{alltt}

  % metadata
  \title{\textsc{Gebruiksvriendelijkheid~van Sociale~Netwerken} \linebreak $Plan~van~Aanpak$}
  \author{\textbf{Kilian Valkhof}\\
  Hogeschool Rotterdam\\
  \\
  \textit{Stagebegeleider:} Sandra Hekkelman\\
  \textit{Tweede begeleider:}\\
  \textit{Bedrijfsbegeleider:} Robert Gaal}

  \date{17 augustus 2009 -- \today}

  \makeindex


  %prettypage
  \hoffset = -0.6in
  \textwidth = 6in

  \usepackage{lastpage}
  \usepackage{fancyhdr}
  \pagestyle{fancy}
  \fancyhead{}
  \fancyfoot{}

  \lhead{}
  \rhead{}
  \lfoot{$Plan~van~Aanpak$}
  \rfoot{\thepage~van \pageref{LastPage}}
  \renewcommand{\headrulewidth}{0.4pt}
  \renewcommand{\footrulewidth}{0.4pt}

  % List items
  \renewcommand{\theenumi}{\roman{enumi}}
  \renewcommand{\labelenumi}{\theenumi}
  \renewcommand{\theenumii}{\alph{enumii}}
  \renewcommand{\labelenumii}{\theenumii}

\begin{document}
  \normalem
  \maketitle

  \newpage
  \tableofcontents

  \newpage
\section{Aanleiding}
In de recente jaren zijn er een tweetal dingen gebeurd: sociale netwerken hebben een explosieve groei doorgemaakt
en gebruiksvriendelijkheid voor websites hebben (terecht) een veel grotere nadruk gekregen dan daarvoor. Hoewel mensen
als Jacob Nielsen en Jesse James Garrett het ons al jaren vertellen, is het pas de afgelopen jaren `normaal' geworden
om ook aandacht te besteden aan gebruiksvriendelijkheid. Veel van de theorie op dit gebied focust zich echter op
informatieve websites (nieuwssites, bedrijfswebsites en weblogs) en minder op de nieuwe vorm van websites: social networks.

Doordat mensen op sociale netwerken heel anders bezig zijn dan op gewone informatieve websites --- er sprake van interactie in plaats van passief lezen --- Denk ik dat dit gevolgen heeft voor de gebruiksvriendelijkheid. Waar gebruiksvriendelijkheid op informatieve websites gaat om het effectief vinden van informatie, is het hoofddoel van sociale netwerken de participatie. Mijn afstudeerbedrijf, \emph{Wakoopa}, is zo'n sociaal netwerk, en is daarom met mij geinteresseerd in het verbeteren van de gebruikersvriendelijkheid van de website en het uitvinden welke factoren voor meer participatie zorgen.

Gedurende mijn minor user experience design heb ik veel aandacht besteed aan gebruiksvriendelijkheid en hoe de gebruiksvriendelijkheid van design
goed vertaalt kan worden naar werkende code. Ik hoop dit door te kunnen zetten bij Wakoopa gedurende mijn afstudeertraject.

\section{Doelstelling}
    Het doel van deze afstudeerstage is uitvinden welke usability factoren invloed hebben op de participatie van gebruikers van sociale netwerken. Ik hoop met een set aanbevelingen te kunnen komen die specifiek gericht zijn op sociale netwerken, en deze aanbevelingen op de site van Wakoopa door te kunnen voeren als casus. Via deze methode kan ik zeggen of de factoren inderdaad invloed hebben, en in welke mate.

\section{Probleemstelling}
Deze afstudeerstage heeft de volgende onderzoeksvraag:
\begin{quotation}
 %\textbf{Hebben sociale netwerken andere eisen op het gebied van gebruiksvriendelijkheid dan de huidige inzichten op webgebied, welke zijn dat en hoe verhoud het sociaal netwerk Wakoopa zich tot deze eisen?}

 \textbf{Hoe kan de participatie op een sociaal netwerk verhoogd worden door middel van usability?}
\end{quotation}


\section{Onderzoeksvragen}
\begin{enumerate}
\item
 \textbf{Wat zeggen andere onderzoeken op het gebied van usability en sociale netwerken?}

Er zijn een aantal onderzoeken over sociale netwerken in het algemeen, en een aantal usability onderzoeken specifiek voor Wakoopa. Ik begin met een analyse van deze onderzoeken, waar mogelijk aan de hand van de Heuristics van Nielssen. Daaruit komen een aantal globaal geldende conclusies en een aantal specifiek voor Wakoopa.

\item
\textbf{Wat vinden gebruikers van het sociaal netwerk Wakoopa op het gebied van usability?}

Door gebruikersonderzoek binnen de leden van Wakoopa doe ik onderzoek naar hun meningen. Dit wil ik kwantitatief d.m.v. A/B-testen, statistiekanalyse en een survey en analyse daarvan, en kwalitatief door middel van interviews en remote/lab tests onderzoeken.

\item
\textbf{Welke verbeteringen vinden we door analyse van data?}
Door te kijken naar statistieken en A/B testen, welke verbeteringen zijn er dan door te voeren?

\item
\textbf{Welke usabilitytechnieken verhogen de participatie op sociale netwerken?}

Hierin beantwoord ik het belangrijkste gedeelte van de hoofdvraag, met conclusies gebaseerd op de vorige hoofdstukken

\item
\textbf{Welke verbeteringen zijn er specifiek voor Wakoopa door te voeren?}

Kijkend naar de conclusies van de vorige hoofdstukken, hoe kan Wakoopa verbeterd worden?

\item
\textbf{Wat zijn de quick wins op het gebied van usability voor sociale netwerken?}

In dit hoofdstuk wil ik een checklist voor sociale netwerken opzetten waarmee andere bedrijven hun sociale netwerk onder de loep kunnen nemen.
\end{enumerate}

\section{Projectactiviteiten en Planning}
Deze afstudeerstage bestaat uit een aantal fasen:

\begin{enumerate}
\item
Analyse van bestaande literatuur
\item
Onderzoek onder gebruikers van Wakoopa
\item
Analyseren onderzoeksresultaten
\item
Vergelijken met bestaande literatuur
\item
Doorvoeren op de website van Wakoopa
\item
Evalueren
\end{enumerate}

Deze activiteiten zal ik in iteraties van 2 weken uitvoeren, waarbij ik de eerste en de laatste week gebruik voor opstarten en afronden. In totaal komt dit uit op 7 iteraties. Na elke iteratie evalueer ik het gedane werk en per iteratie gebruik ik éen onderzoeksmethode. Dit met uitzondering van week 14 -- 15, waar ik de statistiek en A/B tests herhaal. Dit doe ik omdat deze na de verbeteringen uit de andere fasen het snelst nieuwe inzichten zullen geven. De planning inclusief onderzoeksmethodes ziet er zo uit:

\begin{flushleft}
  \begin{enumerate}
    \item Plan van aanpak, vinden referenties, \emph{opstarten}
      \linebreak Week 1
    \item Analyse Literatuur
      \linebreak Week 2 -- 3
    \item Analyse bestaande expert reviews Wakoopa
      \linebreak Week 4 -- 5
    \item Statistiek en A/B testen
      \linebreak Week 6 -- 7
    \item Analyse Enquetering
      \linebreak Week 8 -- 9
    \item Lab tests
      \linebreak Week 10 -- 11
    \item Interviews
      \linebreak Week 12 -- 13
    \item Nogmaals statistiek en A/B testen
      \linebreak week 14 -- 15
    \item \emph{afronding}
      \linebreak week 16
  \end{enumerate}
\end{flushleft}
\section{Projectgrenzen}
Dit project zal zich, hoewel het in beginsel over sociale netwerken in het algemeen gaat, richten op het sociale netwerk Wakoopa. Dit doen we omdat hier naast een expert review ook andere analyse op kan worden toegepast, zoals analyse van de statistieken en interviewen van gebruikers. Ik weerhoud mij dan ook van het doen van uitspraken over andere sociale netwerken specifiek, met uitzondering van de quick wins voor social networks in het algemeen.

In hoeverre we gebruik kunnen maken van technieken zoals A/B testen is nog niet bekend, en zal per item moeten worden besproken met de bedrijfsbegeleider om er voor te zorgen dat het niet nadelig voor Wakoopa werkt.

\section{Producten}
\begin{enumerate}
\item
Een usability onderzoeksverslag met daarin de bevindingen van:
  \begin{enumerate}
  \item
  De geanalyseerde expert reviews
  \item
  De geanalyseerde literatuur
  \item
  De enquete
  \item
  De gebruikersinterviews
  \item
  De lab tests
  \item
  Analyse van de statistieken en A/B testen
  \end{enumerate}
\item
Usabilityverbeteringen doorgevoerd op Wakoopa

De bevindingen uit het onderzoek worden doorgevoerd op de website van Wakoopa en verbeteringen gemeten.
\item
Artikel met aanbevelingen voor social networks

De best werkende aanbevelingen worden gebundeld in een kort "Best practices" artikel
\item
Scriptieverslag

Dit alles wordt beschreven in het scriptieverslag
\end{enumerate}


\section{Kosten en Baten}
De kosten voor dit project bedragen 24 fte, ofwel 80 dagen en 32 uur scriptie schrijven. De baat van dit project is de verbeterde
gebruiksvriendelijkheid van het sociaal netwerk Wakoopa en aanbevelingen voor soortgelijke sociale netwerken.

\section{Risico's}
De volgende risico's zijn aanwezig in dit project:

\begin{itemize}
\item
Werk voor Wakoopa haalt tijd weg van mijn afstuderen.

Dit ondervang ik door duidelijke afspraken met mijn bedrijfsbegeleider qua opdrachten en per week een dag waar ik volledig met schrijven bezig ben.
\item
Het niet volledig kunnen onderzoeken van usability vanwege de benodigde aanpassingen aan de website.

Dit ondervang ik door de grote hoeveelheid onderzoek dat er al is gedaan wat ik kan analyseren, en door ook onderzoeksmethoden te gebruiken die niet van invloed zijn op de website.

\item
Niet toekomen aan het doorvoeren van de wijzigingen vanwege de grootte van het onderzoek.

Door niet met de watervalmethode, maar met een iteratieve methode te werken, zorg ik er voor dat zowel onderzoek als uitvoering aan bod komen.
\end{itemize}

\section{Betrokken partijen}
De betrokken partijen bij deze afstudeerscriptie zijn:

\begin{center}
\begin{tabular}{r|lll}
× & Functie & E-mail & Telefoonnummer\\ \hline
Kilian Valkhof & Afstudeerder & kilian@wakoopa.com & 06 40 96 84 20\\
Robert Gaal & Bedrijfsbegeleider & Robert@wakoopa.com & 06 41 24 23 58\\
Sandra Hekkelman & Stagebegeleider & s.m.hekkelman@hro.nl & × \\
× & Tweede begeleider & × & ×
\end{tabular}
\end{center}

\end{document}

