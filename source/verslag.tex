\documentclass[a4paper, 10pt, pdftex]{report}
  \usepackage[dutch]{babel}
  \usepackage{ulem}
  \usepackage{alltt}
  \usepackage{amssymb}
  \usepackage{graphicx}


  \usepackage{natbib}
  \bibpunct{(}{)}{;}{a}{,}{,}

  % metadata
  \title{\textsc{Participatie op learning networks verhogen door middel van gebruiksvriendelijkheid} \linebreak (werktitel) \linebreak \emph{Projectverslag}}
  \author{\textbf{Kilian Valkhof}\\
  Hogeschool Rotterdam\\
  \textit{studentnr.:} 0783312\\
  \\
  \textit{Afstudeerbegeleider:} Sandra Hekkelman\\
  \textit{Tweede begeleider:}\\
  \\
  \textit{Bedrijf:} Wakoopa bv.\\
  \textit{Bedrijfsbegeleider:} Robert Gaal}

  \date{17 augustus 2009 -- \today}

  \makeindex


  %prettypage
  %\hoffset = -0.6in
  %\textwidth = 6in

  \usepackage{lastpage}
  \usepackage{fancyhdr}
  \pagestyle{fancy}
  \fancyhead{}
  \fancyfoot{}

  \lhead{}
  \rhead{}
  \lfoot{$Projectverslag$}
  \rfoot{\thepage~van \pageref{LastPage}}
  \renewcommand{\headrulewidth}{0.0pt}
  \renewcommand{\footrulewidth}{0.4pt}

  % List items
  \renewcommand{\theenumi}{\roman{enumi}}
  \renewcommand{\labelenumi}{\theenumi}
  \renewcommand{\theenumii}{\alph{enumii}}
  \renewcommand{\labelenumii}{\theenumii}

\begin{document}
  \normalem
  \maketitle

  \newpage
  \chapter*{Samenvatting}
  \addcontentsline{toc}{chapter}{Samenvatting}

  \newpage
  \tableofcontents

  \newpage
  \chapter*{Introductie}
  \addcontentsline{toc}{chapter}{Introductie}
    In de recente jaren zijn er een tweetal dingen gebeurd: sociale netwerken hebben een explosieve groei doorgemaakt
  en gebruiksvriendelijkheid voor websites hebben (terecht) een veel grotere nadruk gekregen dan daarvoor. Hoewel mensen
  als Jacob Nielsen en Jesse James Garrett het ons al jaren vertellen, is het pas de afgelopen jaren `normaal' geworden
  om ook aandacht te besteden aan gebruiksvriendelijkheid. Veel van de theorie op dit gebied focust zich echter op
  informatieve websites (nieuwssites, bedrijfswebsites en weblogs) en minder op de nieuwe vorm van websites: learning networks.

    Een sociaal netwerk staat of valt bij participatie van haar gebruikers. Omdat de interactiviteit van dit soort websites zo'n grote rol speelt, is het belangrijk dat de gebruiker zo min mogelijk drempels tegen komt tijdens het gebruiken van de website. Het weghalen van drempels wordt gebruiksvriendelijkheid genoemd. Dit verslag richt zich op deze gebruiksvriendelijkheid met als doel de interactiviteit beter te laten verlopen en zo de participatie op het sociale netwerk te verhogen.

  Doordat mensen op sociale netwerken heel anders bezig zijn dan op gewone informatieve websites --- er sprake van interactie in plaats van passief lezen --- Denk ik dat dit gevolgen heeft voor de gebruiksvriendelijkheid. Waar gebruiksvriendelijkheid op informatieve websites gaat om het effectief vinden van informatie, is het hoofddoel van sociale netwerken de participatie. Mijn afstudeerbedrijf, \emph{Wakoopa}, is zo'n sociaal netwerk, en is daarom met mij geinteresseerd in het verbeteren van de gebruikersvriendelijkheid van de website en het uitvinden welke factoren voor meer participatie zorgen.

    Gedurende mijn minor user experience design heb ik veel aandacht besteed aan gebruiksvriendelijkheid en hoe de gebruiksvriendelijkheid van design goed vertaalt kan worden naar werkende code. Ik hoop dit door te kunnen zetten bij Wakoopa gedurende mijn afstudeertraject.




    \section{Introductie van Wakoopa}
      Een korte introductie van Wakoopa is voor het verdere verslag van belang, zodat de lezer een duidelijk beeld heeft van Wakoopa als learning network en de mogelijkheden daarvan. Op de About pagina van Wakoopa \citep{Gaal2007} staat de volgende beschrijving:
        \begin{quote} Wakoopa is a learning network that helps people discover the best software, games and web apps on the market. Sign-up, install a small tracker on your desktop and automatically create your online software profile that you can share with friends and the world, also through widgets. Wakoopa keeps you updated about what your contacts are using, and sends you smart recommendations. Games, audio \& video players, instant messengers or office tools: Wakoopa knows what's hot.
        \end{quote}
      Door het installeren van een kleine applicatie op je computer (de tracker), kan Wakoopa bijhouden welke applicaties je allemaal op je computer gebruikt. Deze gegevens worden in een online profiel weergegeven. Daarnaast kunnen gebruikers hun mening geven over de applicaties die zij gebruiken. Dit wordt gecombineerd met een sociaal aspect van het leggen van contacten, het maken van teams, het behalen van punten en het \emph{raten} van applicaties.

        \subsection{Wakoopa als learning network}
        In hun paper \emph{Functionality for learning networks: lessons learned from social web application} noemen \citeauthor{Berlanga2007} een aantal kenmerken van sociale netwerken die ook \emph{learning networks} zijn. In plaats van het maken van contacten is het focuspunt van deze sociale netwerken objecten. Een voorbeeld hiervan is Flickr, een sociaal network rondom foto's. Op Wakoopa zijn deze objecten de applicaties die je gebruikt. Om vast te stellen of Wakoopa onder dezelfde categorie valt en om een overzicht te geven van de functionaliteit die Wakoopa biedt, zullen we in tabellen \ref{tab:functies} \ref{tab:acties} en \ref{tab:metaacties} \label{learningnetwork} Wakoopa vergelijken met de kenmerken die \citeauthor{Berlanga2007} hebben opgesteld.

         De drie door \citeauthor{Berlanga2007}  onderzochte learning networks (Delicious, Youtube en Flickr) bevatten niet \emph{alle} omschreven functionaliteit, maar worden hoe dan ook als \emph{learning networks} omschreven. Wakoopa voldoet in deze tabellen ook niet aan alle vereisten, maar zit qua functionaliteit op vergelijkbare hoogte met Youtube en Flickr (waarbij Delicious minder functionaliteit bied). We kunnen Wakoopa dus als learning network beschouwen.

        \begin{table}[ht]
        \centering
        \begin{tabular}{r|l}
          × & Wakoopa \\ \hline
          Profile & \checkmark \\
          Contacts & \checkmark \\
          Resources & \\
          Tagging & \checkmark
        \end{tabular}
        \caption{Self-management functionality}
        \label{tab:functies}
        \end{table}
        \begin{table}[ht]
        \centering
        \begin{tabular}{r|l}
        \label{acties}
          × & Wakoopa \\ \hline
          Comment & \checkmark \\
          Recommend & \\
          Copy & \\
          Subscribe & \checkmark \\
          Add as favourite & \checkmark \\
          Rate & \checkmark \\
          Related resources & \checkmark \\
          Search & Software, Users, Teams, Developers
        \end{tabular}
        \caption{Self-organisation functionality}
        \label{tab:acties}
        \end{table}
        \begin{table}[ht]
        \centering
        \begin{tabular}{r|l}
        \label{metaacties}
          Markeer\ldots & Wakoopa \\ \hline
          Resources as offensive & \checkmark \\
          Communities as offensive & \\
          Private and public resources & \checkmark \\
          Private and public communities/groups & \checkmark
        \end{tabular}
        \caption{Self-regulation functionality}
        \label{tab:metaacties}
        \end{table}

  \newpage
  \chapter*{Voorwoord}
  \addcontentsline{toc}{chapter}{Voorwoord}

  \newpage
  \chapter*{Doelstelling}
  \addcontentsline{toc}{chapter}{Doelstelling}
    Het doel van deze afstudeerstage is uitvinden welke usability factoren invloed hebben op de participatie van gebruikers van sociale netwerken. Ik hoop met een set aanbevelingen te kunnen komen die specifiek gericht zijn op sociale netwerken, en deze aanbevelingen op de site van Wakoopa door te kunnen voeren als casus. Via deze methode kan ik zeggen of de factoren inderdaad invloed hebben, en in welke mate.

  \chapter*{Probleemstelling}
  \addcontentsline{toc}{chapter}{Probleemstelling}
    Deze afstudeerstage heeft de volgende onderzoeksvraag:
    \begin{quotation}
     \textbf{Hoe kan de participatie op een learning network verhoogd worden door middel van usability?}
    \end{quotation}

  \newpage
  \chapter{Wat zeggen andere onderzoeken op het gebied van usability en sociale netwerken?}
    \label{researchchapter}
    \newpage

    In dit hoofdstuk bekijken we wat er blijkt uit onderzoeken uitgevoerd door anderen op het gebied van usability en sociale netwerken, en hoe dit tot verhoging van de participatie leidt. We onderzoeken twee soorten bronnen: Papers die in het algemeen op sociale netwerken ingaan, en een drietal onderzoeken specifiek voor Wakoopa uitgevoerd.

      De bevindingen van deze onderzoeken zullen waar mogelijk door middel van A/B testen doorgevoerd worden op Wakoopa. De resultaten hiervan zijn te vinden in Hoofdstuk \ref{datachapter}.

    \section{Externe onderzoeken}
      \subsection{\cite{Beenen2004}}

      In \emph{Using social psychology to motivate contributions to online communities} onderzoeken \citeauthor{Beenen2004} welke factoren en stimulansen bijdragen aan meer participatie van gebruikers, in hun case die van een filmsite. Door middel van een onderzoek met doelen voor gebruikers, waarbij ze de bewoording aanpasten, kwamen ze tot de conclusie dat, wanneer je aan een gebruiker duidelijk maakt hoe uniek ze zijn, ze dan veel meer zullen participeren op de website. Daarentegen is het heel lastig ze te motiveren. Enkel het noemen van voordelen om te participeren zorgt er volgens hun onderzoek voor dat mensen dat minder snel zullen doen. Een mogelijke verklaring die ze hiervoor geven is dat, wanneer mensen vertelt wordt dat ze iets moeten doen, ze minder snel geneigd zijn dat ook daadwerkelijk te doen.

      Volgens het onderzoek werkt dit zo, omdat mensen gestimuleerd worden door intrensieke motivatie, maar juist minder snel zullen participeren wanneer ze een extrensieke motievatie wordt gegeven. De overkoepelende conclusie is dat je gebruikers moet tonen hoe uniek hun bijdragen zijn, zonder dat je daarbij vermeld wat de voordelen van deze bijdragen zijn.

      Bij Wakoopa kan dit toegepast worden door bijvoorbeeld de \emph{call to actions} te bekijken, en ons af te vragen of dit slechts het noemen van een voordeel is, of laat zien op welke manier een gebruiker zich hiermee kan onderscheiden.

     \subsection{\cite{Sohn2005}}

      In \emph{Dimensions of interactivity: Differential effects of social and psychological factors} onderzoeken \citeauthor{Sohn2005} uit welke componenten interactiviteit bestaat, en welke eigenschappen of omgevingen van invloed zijn op deze componenten. Uit hun onderzoek blijkt dat interactie bestaat uit een drietal componenten:
        \begin{enumerate}
          \item Controle
          \item Reactiekwaliteit
          \item Werkbaarheid van de interactie
        \end{enumerate}
      Na analyse van de eigenschappen van proefpersonen en hun netwerk, kwamen er vier factoren uit die invloed hadden op de componenten van interactiviteit. Deze zijn:
        \begin{description}
          \item[Need for cognition]
            Need for cognition is een term die gebruikt wordt om aan te geven hoe leergierig je bent.
          \item[Web usage time]
            De tijd die je op het web spendeert.
          \item[Communication direction]
            de richting van de communicatie, dit kan naar de proefpersoon zijn, maar de proefpersoon kan tegen met andere mensen uit zijn netwerk praten.
          \item[Network density]
            Dit is de mate waarin de sociale relaties van de proefpersoon ook connecties met elkaar hebben. Met andere woorden: hoeveel van jouw vrienden kennen anderen van jouw vrienden?
        \end{description}
        Van deze vier factoren waren \emph{need for cognition} en \emph{web usage time} de meest significante indicatoren voor de mate waarin de gebruiker interactiviteit ervaart. \emph{Need for cognition} was van importantie bij alle drie de componenten. \emph{Web usage time} enkel op de werkbaarheid van de interactie. \emph{Communition direction} en \emph{Network density} hebben beide invloed op de reactiekwaliteit.

        \paragraph{}
        Voor learning networks in het algemeen betekent dit een aantal dingen:
        \begin{itemize}
          \item Maak het gemakkelijk om connecties met anderen te leggen (network density verhogen)
          \item Zorg voor stimulansen die de nieuwschierigheid van gebruikers opwekken (need for cognition)
          \item Zorg voor passieve berichtgeving van je netwerk, bijvoorbeeld wanneer connecties hun profiel wijzigen (communication directions)
          \item Zorg ervoor dat gebruikers langere tijd iets op je site te doen hebben of redenen hebben om terug te keren (web usage time)
        \end{itemize}
        Een aantal van deze punten worden in het geval van Wakoopa al gebruikt. Het is gemakkelijk om nieuwe connecties te leggen (dit kan via een klik op iemands profiel) en we stimuleren dit door aan te geven welke gebruikers op jou lijken. We laten mensen weten wanneer hun vrienden nieuwe applicaties gebruiken, een review schrijven, een level omhoog gaan of iets op een teampagina schrijven. Dit punt wordt ook ondersteund door onderzoek van \cite{Berlanga2007}. We stimuleren nieuwsgierigheid door het puntensysteem, waarbij gebruikers meer punten verdienen door meer software te gebruiken. Een punt waar verbetering te behalen valt is het langer vasthouden van bezoekers. Wakoopa kan dit verbeteren door mensen meer acties op de site uit te laten voeren, en interessante(re) statistieken weer te geven op profielen.

    \subsection{\cite{Brouns2008}}

    In \emph{Personal profiles: Facilitating participation in Learning Networks} onderzoeken \citeauthor{Brouns2008} op welke manieren bestaande learning networks de participatie verhogen. Ze onderzochten hiervoor Schoolbank, Schoolpagina, Hyves, Facebook, Myspace en LinkedIn. De nadruk werd hierbij gelegd op de manieren hoe profielen werden aangemaakt en hoe de learning networks het compleet maken van deze profielen stimuleerden.

    Een methode die volgens de onderzoekers goed werkte was het laten zien van een progressiemeter. Dit wordt door LinkedIn toegepast. Iedere actie die een persoon nog moet uitvoeren om zijn of haar profiel compleet te maken zit gekoppelt aan een bepaald percentage. Wanneer je een bepaalde actie nog niet hebt gedaan, is de balk nog niet vol, en staat er onder de balk in een tekstlink de eerstvolgende actie. Deze methode is (na het schrijven van deze paper) overgenomen door Facebook, die eenzelfde soort progressiemeter laat zien na het aanmelden en tijdens het aanmaken van een profiel.

    Naast het invullen van een profiel werd er ook gekeken hoe gebruikers tijdens het proces van aanmelden en invullen van gegegevens gestimuleerd konden worden. De twee punten die hieruit naar voren kwamen is dat het duidelijk moet zijn welk doel het invullen van een bepaald invoerveld heeft, en waarom het belangrijk is om de invoervelden waarheidsgetrouw in te vullen. Voorbeelden die door \citeauthor{Brouns2008} worden gegeven zijn: het goed lopen van het gehele systeem; het correct kunnen vinden van contacten en informatie; het krijgen van goede aanbevelingen.

    Net als \citet{Berlanga2007} en \citet{Sohn2005} onderstrepen \citeauthor{Brouns2008} het belang van gebruikers op de hoogte brengen van wijzigingen aan de profielen van contacten, en geven aan dat dit een methode is om gebruikers ``geinteresseerd en gemotiveerd'' te houden.

Momenteel toont Wakoopa enkel een melding dat een profiel nog niet compleet is, maar geeft niet aan wat de vervolgstappen zijn. De implementatie van een progressiemeter kan er voor zorgen dat meer gebruikers hun profiel invullen.

    \subsection{\cite{Editorial2008}}

    Smashing Magazine, een bekende weblog over web development technieken, heeft in \emph{Web Form Design Patterns: Sign-Up Forms} onderzoek gedaan naar de aanmeldformulieren van honderd learning networking sites\footnote{http://media2.smashingmagazine.com/images/web-form-design-patterns/urls.html, geraadpleegd op 9 september 2009}. Een van de opmerkelijke feiten was dat in 43\% van de websites, de sign-up link rechtsboven stond. Op Wakoopa staat er op de homepagina geen sign-up link bovenaan, op overige pagina's staat hij links bovenaan. Door middel van A/B testen zullen we kijken of de plaatsing effect heeft op het aantal clicks.

    \subsection{\cite{Sloep2009}}

    in \emph{From lurker to active participant} onderzoeken \citeauthor{Sloep2009} hoe je passieve gebruikers (``lurkers'') kan motiveren om actief te participeren in een community. In hun paper gaan ze uit van een fictieve community, en hebben daar een aantal persona's voor gemaakt. Participatie op sociale netwerken ontstaat onder een een viertal voorwaarden nodig:
    \begin{itemize}
    \item Gebruikers moeten een persistente identiteit hebben. Dit hoeft geen echte naam zijn, maar kan ook een pseudoniem zijn.
    \item Er mag geen vastgesteld einde zijn, zoals een einddoel.
    \item Probeer ervoor te zorgen dat iedere participatie als even waardevol wordt beschouwd. Latere participaties mogen minder waardevol zijn, zolang de daling maar gelimiteerd blijft.
    \item Zorg ervoor dat een gebruiker zijn prestaties aan anderen kan laten zien.
  \end{itemize}
    Wanneer deze voorwaarden voldaan zijn, zal volgens \citeauthor{Sloep2009} participatie voornamelijk uit zichzelf ontstaan. Kijkend naar Wakoopa zien we dat deze voorwaarden inderdaad voldaan zijn, en dat vanuit deze voorwaarden Wakoopa uit zichzelf zal aanzetten tot participatie.

  \subsection{\cite{Wroblewski2009}}
    in \emph{Inline Validation in Web Forms} onderzoekt \citeauthor{Wroblewski2009} welke methode van inline validatie het beste werken bij formulieren. Inline validatie is het controleren op juistheid van de input, op het moment dat de gebruiker een actie uitvoert. Dit is anders dan de traditionele methode, waarbij de gebruiker eerst de pagina moet opsturen en deze pas na herladen aangeeft of zij het formulier correct heeft ingevuld. Dit onderzoek is relevant voor sociale netwerken, omdat deze meer interactiviteit bieden en daardoor meer input verwachten van de gebruiker. Wanneer dit sneller en beter verloopt, en de gebruiker het idee heeft controle te hebben over de interactie (zoals beschreven in \cite{Beenen2004}), zal de participatie verhogen. In dit onderzoek heeft \citeauthor{Wroblewski2009} twee\"{e}ntwintig 'gemiddelde gebruikers'  (als definitie wordt later aangegeven dat het niet om mensen die blind kunnen typen gaat) met een aantal verschillende formulieren laten werken, en met een aantal usability-onderzoekstechnieken (eye-tracking, lab-test en nabespreking) gekeken welke variaties het beste werkte.

    Vooropgesteld kwam de onderzoeker er achter  dat iedere vorm van inline validatie er voor zorgt dat gebruikers sneller en met minder fouten een formulier door kunnen lopen. Uit het onderzoek bleek dat er twee soorten vragen waren; vragen waar een gebruiker niet over na hoeft te denken, zoals zijn voornaam, en vragen waarbij een gebruiker wel moest nadenken, zoals het kiezen van een wachtwoord. In de eerste situatie voegt inline validatie weinig toe, maar in de tweede situatie zorgt het voor een aanzienlijke verbetering in het doorlopen van het formulier, alsook het maken van minder fouten.

    Belangrijk is waneer je de validatie laat zien. Is dit al van te voren, of tijdens het typen, dan werkt dit verwarrend voor de gebruiker. De meest effectieve validatie is het weergeven van een bericht zodra een gebruikler klaar is met het invullen van een formulierveld. De verklaring die de onderzoeker hier voor had was dat, wanneer er tijdens het typen al een bericht zichbaar is, de gebruiker tussen iedere getypte letter kijkt of het ``al goed is''. Dit heeft ook effect op waar je een bericht laat zien. Pas je inline validatie in, dan moet er bij ieder invoerveld een bericht komen, anders breng je je gebruiker in verwarring.

    Naast het weergeven van een bericht testte de onderzoeker ook of het permanent weergeven, of het langzaam laten wegfaden van een bericht beter was. Omdat niet iedere gebruiker continue naar het scherm keek, kwamen zij tot de conclusie dat een persistene berichtgeving beter was.

  \section{Interne onderzoeken bij Wakoopa}
    Sinds het online plaatsen het nieuwe design van Wakoopa in 2008 zijn er een drietal usability onderzoeken uitgevoerd: \citet{Timmerman2008, Hoekman2008, Alfrink2008}. De bevindingen van deze usabilityonderzoeken en op welke manier ze momenteel op Wakoopa van toepassing zijn worden hieronder beschreven.

    \subsection{Usability Review \citet{Alfrink2008}}
    Tijdens de ontwikkeling van het nieuwe ontworp is er door Leapfrog een expert review van het toen in ontwikkeling zijnde ontwerp. Bij deze expert review is gebruikt van een aantal heuristics, zoals die van Jacob Nielsen \footnote{http://www.useit.com/papers/heuristic/heuristics\_list.html, geraadpleegd op 9 september 2009} en die van Steven Kruger uit zijn boek \emph{Don't make me think}. Deze laatste staan niet online beschreven, en zijn daarom hieronder opnieuw geprint:

      \begin{enumerate}
        \item Create pages that are self-evident, or at least self-explanatory
        \item Create a clear visual hierarchy
        \item Take advantage of conventions, only innovate when you know you have a better idea
        \item Break pages up into clearly defined areas
        \item Make it obvious what's clickable
        \item Assume everything is visual noise until proven otherwise
        \item Make choices mindless
        \item Omit needless words
      \end{enumerate}

    In het onderzoek van \citeauthor{Alfrink2008} worden veel detailpunten besproken, met veel nadruk op het verhogen van gebruik. Wanneer je dit vertaalt naar globale richtlijnen komen er een aantal punten uit. Zo kan je gebruikers best op een directere manier om participatie, zoals het schrijven van een review, vragen, en hier kan je eventueel punten tegenover stellen. Hetzelfde proces wordt beschreven voor direct na het inloggen. Wat moet een gebruiker nu doen? Door middel van uitgebreidere begeleiding maak je het de gebruiker gemakkelijker, in een stadium waar de gebruiker nog niet bekend is met het systeem. Dit kan ook later door bij verschillende onderdelen op de site duidelijk de waarde van een functie aan te geven. Bijvoorbeeld het taggen van items of het aangeven waarom je bepaalde aanbevelingen krijgt.

    Soorgelijke doortastende dingen zijn te doen met andere punten van een site. Zo kan je bij zoekfunctionaliteit bijvoorbeeld voorspellen waar de gebruiker naar wilt zoeken afhankelijk van het soort pagina waar hij of zij op zitten. Wanneer een gebruiker op een andere gebruikerspagina zit, zal hij of zij waarschijnlijk naar gebruikers zoeken, terwijl wanneer je op een objectpagina waarschijnlijk naar andere object op zoek bent. Op een globale overzichtspagina kan je ook persoonlijke informatie kwijt, zoals bij categorie\"en de applicaties die jij in die categorie gebruikt.

    \subsection{Usability Review \citet{Hoekman2008}}
    In tegenstelling tot het onderzoek van \citeauthor{Alfrink2008} richt het usabilityonderzoek van Miskeeto zich meer op de globale indeling van de pagina's en de navigatie hierop. De nadruk wordt gelegd op een homepage die zeer duidelijk de voordelen (en expliciet niet de \emph{functionaliteit}) uitlegt, en dit in een duidelijk visueel blok zet. \citeauthor{Hoekman2008} Maken een punt voor een abstracter niveau van navigatie, waar dit in drie delen wordt opgedeeld: website-brede navigatie; secundaire navigatie en object navigatie. Dit laatste gaat om de pagina's die bij een bepaald object horen (zoals bijvoorbeeld een pagina met alle tags voor een object) Door deze strict gescheiden te houden, zorg je ervoor dat de gebruiker niet per se hoeft te onthouden waar bepaalde functionaliteit zit, maar dit kan afleiden aan het type functionaliteit.

    Dit idee wordt ook gebruikt als tip voor andere delen van een site. Door specifieke blokken een gelijke kleur te geven (zoals bijvoorbeeld \emph{geel} voor \emph{persoonlijk}) cree\"er je een snel overzicht van welke delen van de pagina waarop van toepassing zijn. Dit moet echter wel zeer consistent zijn doorgevoerd, omdat het anders de bezoeker zal verwarren.

    \subsection{Usability Review \citet{Timmerman2008}}
    \citeauthor{Timmerman2008} van Usarchy heeft in zijn review veel aandacht voor de analyse van gegevens en algemeen gebruikte usabilitytechnieken. Volgens hem is het erg belangrijk om te beginnen met het maken van persona's. Dit zijn fictieve personen die jouw learning network gebruiken. Voor elk van de verschillende doelgroepen maak je er eentje. Door deze persona's zo echt mogelijk te maken (inclusief naam, foto, hobbies) kan je ze gebruiken om bij nieuwe functionaliteit te kijken voor welke persona, en dus welke doelgroep, je het maakt.

    Verder maakt \citeauthor{Timmerman2008} de case om op de site behoeftegericht te werken. Door teksten op zo'n manier aan te passen dat ze de behoefte voor een gebruiker vervullen, zorg je ervoor dat deze gebruikers actiever zullen participeren.

    Het is ook belangrijk om de site te testen, bijvoorbeeld door middel van A/B testen, het analyseren van clickmaps en door het maken van `sales' funnels in een statistiekprogramma. Via deze methoden kan je uitvinden wat momenteel de knelpunten op een learning network zijn, en hoe deze te zijn verbeteren.


  \newpage
  \chapter{Wat vinden gebruikers van het sociaal netwerk Wakoopa op het gebied van usability?}
    \newpage

  \newpage
  \chapter{Analyse van de data}
    \label{datachapter}
    \newpage
    \section{A/B testing}
    A/B testing, of multivariate testing, is een methode om twee (A/B) of meerdere (multivariate) variaties op een pagina of lay-out te testen, door deze gedurende een periode willekeurig onder bezoekers te verdelen. Bezoeker \emph{A} krijgt bijvoorbeeld variatie 1 te zien, en bezoeker \emph{B} krijgt variatie 2 te zien. Vervolgens kijk je welke gebruiker sneller of vaker op de door jouw gekozen link klikt of actie uitvoerd. Wanneer je dit met een groot aantal bezoekers gedurende een langere tijd doet, kan je hier statistische analyse op uitvoeren.

   Het ontwikkelplatform wat Wakoopa gebruikt, Ruby on Rails, heeft door middel van een plugin de optie om A/B testen uit te voeren. Deze automatiseert het verdelen van de verschillende opties tussen bezoekers en houd per variatie bij hoe vaak de geteste links of functionaliteit aangeklikt wordt.

    Naar aanleiding van de in Hoofdstuk \ref{researchchapter} genoemde onderzoeken hebben we een aantal A/B tests uitgevoerd, die hieronder beschreven staan:

    \subsection{Plaats van de sign-up link op landing pages}
      In tegenstelling tot de homepagina hebben onze landing pages (Pagina's waar bezoekers via zoekmachines op terecht komen) wel een sign-up link in de header. Momenteel staat deze in de linkerbovenhoek. In deze A/B test bekijken we of een variate waarin deze in de rechterbovenhoek staat, tot meer clicks leidt dan wanneer deze in de linkerbovenhoek staat.

      [resultaten]

    \subsection{Het benoemen van de mate waarin een profiel is ingevuld}
      Wakoopa geeft gebruikers al een berichtje na het inloggen wanneer een profiel nog niet volledig is ingevuld. Uit onderzoek van \cite{Brouns2008} blijkt dat het effectiever is om hier een vervolgstap of een progressiemeter neer te zetten. In deze test bekijken we een viertal variaties: De huidige berichtgeving, een berichtgeving met welk eerstvolgende veld ze nog moeten invullen (bv. Bio), een berichtgeving met een progressiemeter, en een berichtgeving met zowel een progressiemeter als wel eerstvolgende veld ingevuld moet worden.

  \newpage
  \chapter{Usabilitytechnieken die de participatie op een learning network verhogen}


  \newpage
  \chapter{Welke verbeteringen zijn er specifiek voor Wakoopa door te voeren?}
    \newpage

  \newpage
  \chapter{De quick wins om participatie te verhogen op learning networks}
    \newpage

  \newpage
  \chapter*{Conclusie \& Aanbevelingen}
  \addcontentsline{toc}{chapter}{Conclusie \& Aanbevelingen}

  \newpage
  \chapter*{Discussie}
  \addcontentsline{toc}{chapter}{Discussie}

  \newpage
  \chapter*{Verklarende woordenlijst}
  \addcontentsline{toc}{chapter}{Verklarende woordenlijst}

  \newpage
  \bibliography{../references/referenties}
  \bibliographystyle{plainnat}
  \addcontentsline{toc}{chapter}{Bibliografie}


\end{document}

