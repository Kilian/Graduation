\chapter{Interviews}
\label{interviewappendix}
Er zijn drie usability experts geinterviewd over hun ervaringen met learning networks en usability en hun visie daarop. Voor dit interview is een losse structuur aangehouden, met de volgende vragen als leidraad:

\begin{itemize}
  \item Welke learning networks gebruik je zelf?
  \item Welke learning networks zijn goede voorbeelden van usability en welke minder?

  \item Welke struikelblokken kom je tegen op learning networks's tijdens onderzoeken
  \item Hoe ga je daar mee om tijdens het beoordelen of ontwerpen van een learning networks, wat zijn oplossingen?

  \item Zijn er recente stijl/designontwikkelijken, op usability gebied (zoals bijvoorbeeld google suggest) die je goed vind
  \item En slecht?

  \item Welke onderzoeksmethodieken zijn jouw favoriet? Als je veel budget hebt, en als je weinig budget hebt?
  \item Is er een methode die je verrast heeft toen je hem voor het eerst gebruikte?

  \item Welke methodieken vind jij handig voor learning networks?
  \item Welke vind je minder handig?

  \item Waar zie jij het vakgebied in de toekomst heengaan, wat zijn opkomende technieken volgens jou?
\end{itemize}

Er is gekozen voor deze losse structuur omdat usability een heel breed gebied is. Iedere expert heeft een specialisatie en door de vrijheid te laten kan de expert zijn expertise toepassen op de gestelde vragen.

\section{Bas Bakker}
Bas Bakker is een zelfstandig ondernemer die is afgestudeerd op het ontwerpen en bouwen van een learning network, `goedkado.nl'. Dit is een website waarop mensen kadotips voor vrienden kunnen krijgen aan de hand van integratie met bijvoorbeeld hyves. Bas maakt geregeld expert reviews voor bestaande websites.

\paragraph{}
\textbf{Kilian:} Welke learning networks ben je zelf op actief?

\textbf{Bas:} Ik heb net geleerd dat hyves geen learning network is. Ik gebruik linkedin, flickr, en youtube gebruik ik wel eens, maar wie doet dat nou niet? Twitter gebruik ik ook, maar ik geloof niet dat dat er onder valt. Ning, ik weet niet of dat er onder valt, dat ligt eraan welk ning netwerk natuurlijk. Dit waren zowiezo wel de hoofddingen die ik gebruik.

\textbf{Kilian:} Heb je misschien voorbeelden van hoe Ning ingedeeld kan worden?

\textbf{Bas:} Er zijn twee verschillende voorbeelden. De een is vooral netwerken en de andere is inderdaad meer community waar je ook leert. Wat je veel ziet is dat nings worden opgericht op basis van een offline netwerk. mensen die een community maken om in contact met elkaar te blijven. Opencoffee leiden is daar een voorbeeld van, maar daar gebeurt niet zoveel. Een ander voorbeeld is bijvoorbeeld die van de zaanstreek, daar laten ze zien hoe mooi de zaanstreek is, plaatsen er foto's op en er zijn discussies. De dingen waar ik Ning het meest voor gebruik beginnen offline en gaan dan ook online.

\textbf{Kilian:} Als je kijkt naar ning, of learning networks in het algemeen, zijn er dan gelijk al dingen die je zelf anders zou doen op het gebied van usability?

\textbf{Bas:} Daar moet ik even over denken. Ik vind het concept van ning best slim, zakelijk, en voor niks kan je een netwerk opzetten. Qua usability laat het te wensen over. Het gaat nu vooral over "nu en wat er nog komen gaat". Een gebeurtenis verdwijnt uiteindelijk en dan moet je gaan graven om dat terug te vinden. Dus om een community in te gaan is vrij moeilijk. Dat zou beter kunnen. En de algemene usability...Je moet het een keertje doen, en dan ga je het pas begrijpen. Dat zie je bij heel veel communities, Je komt er, en dan is er geen doel voor jou. Het is vrijheid blijheid, en je kan alle kanten op. Mensen verliezen zich daar in. Dat zie je bij twitter. Mensen zitten daar zonder dat ze weten wat ze aan het doen zijn. Daarnaast draagt het ook niet altijd bij aan het netwerk.

Een voorbeeld daarvan is bij meetup, een website om bij conferenties te zeggen dat je ook komt, de grootste groep daar is werkeloze huismoeders die thuis zitten en met elkaar in contact willen komen. Ieder sociaal netwerk zorgt ervoor dat mensen zelf hun doel kunnen bepalen. Als ze dat hebben gevonden gaan mensen er wel heen, alleen wanneer ze beginnen denken veel mensen "wat moet ik hier mee?". Bij ning zie je dat als je er komt, dan weet je niet wat er in het verleden is gebeurd, en dan weet je niet wat je daar mee moet. Wat je ziet is dat mensen zich aanmelden en ze worden actief, of ze komen nooit meer terug.

\textbf{Kilian:} Ning is dus goed als ondersteunend van een offline community, dus wat je meer moet hebben is dat mensen in het verleden dingen hebben gedaan en daar over verder praten en dingen mee kunnen doen in plaats van enkel in de toekomst kijken.

\textbf{Bas:} Op zich zijn er ook wel genoeg netwerken op Ning die enkel online zijn. Dan nog is het heel moeilijk om er in te graven. Het is een forum, en als je eenmaal 100 discussies verder bent weet je niet wat er heeft gespeeld. Als je er al een tijdje bent is dat geen probleem, maar als nieuwe gebruiker moet je eerst een hoop onderzoek doen naar wat er al is gebeurd. Als je dat niet doet wordt je direct raar aangekijken. Ning is eigelijk net een stap verder dan een forum want het kan iets meer, maar het is slechts éen stapje verder. Het mist het directe doel.

\textbf{Kilian:} Is dat iets wat jij als je zelf iets zou bouwen, wat jij zou doen? Als je op de site komt dat je gelijk aangeeft "dit is het doel"?

\textbf{Bas:} Dat ligt eraan wat het doel is van wat je bouwt. Bij een klant die een sociaal netwerk wilt maken zoals Hyves, dan zou ik dat niet doen. Maar als je een netwerk rondom initiatieven maakt, dan moet je wel neerzetten wat het initiatief is. Bijvoorbeeld bij Flickr is het heel duidelijk dat het om foto's gaat. Als het echt om éen ding gaat, als learning network, dan moet je dat wel neerzetten ja. Bij hyves kan je het niet in drie zinnen neerzetten omdat er zoveel manieren zijn.

\textbf{Kilian:} Hoe zou dat doel gepresenteerd moeten worden? Als die drie regels uitleg, of een complete tutorial?

\textbf{Bas:} Ja, ik zou het allebij doen. Het moet zowiezo in éen oogopslag duidelijk zijn. Voor de "luie bezoeker" moet het gelijk duidelijk zijn, en als iemand meer informatie wilt, dan moet het ook beschikbaar zijn, dan moet er ook die tutorial zijn. Maar als die eerste doelen weg zijn dan moeten mensen dus een tutorial doorlopen om te weten waar het over gaat. Bijvoorbeeld Google wave, dat kwam uit en toen moest je een film van een uur kijken. Ik heb het niet gedaan. Toen ging ik google wave kijken en ik snapte er niks van. Nu heb ik er een beetje mee gespeeld en is het wel leuk, maar dat is een beetje dat vrijheid blijheid. Er zit geen doel achter, en dan moet je veel uitleggen. Maar ik denk op het moment dat je zegt "dit zijn de doelen" dan is dat helder, en daarna met meer informatie als mensen dat willen.

\textbf{Kilian:} Een recente trend om een learning network in een video uit te leggen. Wat vind jij daar van?

\textbf{Bas:} Als je kijkt hoe dat werkt met leren, dan leer je eerst iets, dat commit je en dan ga je het doen, en dan ga je weer leren, commiten, doen en zo voorts. Dus je zal het eerst moeten leren. Qua design kan je dingen zo duidelijk mogelijk maken. Bijvoorbeeld met een duidelijke call to action. Ook hier is beide weer beter. Een video maken kost gewoon tijd, maar het is wel duidelijk. Mensen focussen zich op een video, en op een site niet. Als je een video aanzet, zijn mensen daarop gefocussed terwijl ze op een site overal kijken. Alleen mensen verplichten is weer geen goed idee. Je moet een balans vinden tussen een intuitief design, en daarnaast een video om het uit te leggen. Je hebt altijd de helft blijft en de helft gaat wel weg.

\textbf{Kilian:} Dus zoveel mogelijk verschillende manieren werkt volgens jou het best?

\textbf{Bas:} nouja, niet zoveel mogelijk, maar wel meerdere.

\textbf{Kilian:} De site van jouw afstudeerproject, goedkado, is ook een soort learning network, ben je daar ook problemen tegen gekomen qua usability?

\textbf{Bas:} Ja, daar moet ik even over denken. Een van de dingen was dat de navigatie aan de linkerkant stond en totaal over het hoofd werd gezien. Ik had een aanname gedaan door dit over te nemen van bol.com en amazon die dat ook allebei doen. Mensen zagen dat niet. Dat was gemakkelijk op te lossen door de navigatie bovenaan overdwars te zetten. Een ander punt is dat de labels van navigatie en de structuur van de site. Je doet aannames over wat je doet en koppelt daar namen aan. Bijvoorbeeld 'kadotips'. We geven je kadotips voor vrienden, maar het is heel onduidelijk wat dat betekend omdat mensen daar niet bekend mee zijn. Ze weten niet voor wie, en wat daar achter zit. Ze wilden zoeken bij kado's, en niet bij kadotips. Jouw gedachten en aannames sluiten soms niet aan bij wat de meeste gebruikers vinden. Er zit ook een deel uitleggen bij, mensen wisten niet waarvoor die kadotips dan waren, en hoe ze gemaakt werden. Ze willen weten hoe het werkt. Die beschrijvingen zou er ook bij moeten.

\textbf{Kilian:} Het gaat dus erg om de uitleg, en mensen willen het begrijpen en daar moet je in voorzien.

\textbf{Bas:} Ja dat klopt. Je probeert een probleem op te lossen, maar in de loop van de jaren zijn mensen daar wantrouwig naar geworden. Ze geloven niet dat dat vanzelf gaat, dus ze willen weten hoe het probleem wordt opgelost. Niet technisch natuurlijk, maar gewoon in normale taal, dat inzicht.

\textbf{Kilian:} Je zegt dus eigelijk dat in plaats van aangeven dat er een probleem is, wat vaak gebeurt, dat je ook moet uitleggen hoe je dat op kan lossen?

\textbf{Bas:} Dus eigelijk, dat je het probleem oplost is niet genoeg, je moet het oplossen. Het heeft vooral met vertrouwen te maken. Dit zie je ook bij de aanbevelingen op bol.com en amazon. Ik bekijk dat altijd, maar het is wel interessant om te weten hoe die gemaakt worden. Misschien past dat wel helemaal niet bij mij. Ik kan zelf wel een selectie van interessante aanbevelingen maken, maar het is ook interessant om te weten wat voor soort mensen dan precies die aanbevelingen kochten. Met dat kan ik gelijk een selectie maken van wat wel en niet interessant is voor mij. Het komt dus terug op dat je het vertrouwen moet krijgen. Je moet kijken naar de twee soort personen. Je hebt mensen die snel keuzes maken, en mensen die meer de diepte in willen en snel kunnen schakelen.

\textbf{Kilian:} Dat klopt ook wel met een artikel wat ik voor mij scriptie hebt gebruikt over persona's, daar zijn vier typen mensen. Er zijn mensen die snel gebaseerd op feiten of snel gebaseerd op emotie en mensen die sloom gebaseerd op feiten of snel gebaseerd op emotie beslissen. Dat klopt ook wel met jouw indeling volgens mij.

\textbf{Bas:} Ik denk inderdaad dat het zo werkt. Iedereen is natuurlijk uniek in zijn denken. Iedereen heeft zijn eigen perspectief. Maar op het moment dat jij een van de twee groepen niet faciliteerd, dan verlies je ze. Specialisten die willen de feiten en weten hoe ze dat kunnen controleren, en de generalisten willen juist snel kunnen zien waar het over gaat. Daar moet je een balans tussen vinden.

\textbf{Kilian:} Op Wakoopa staan ook recommendations, en daar is ook veel behoefte aan uitleg. Volgens mij komt dat omdat er vooral de specialisten op wakoopa zitten, en die willen volgens mij vaak weten hoe het precies zit. Dat is lastig omdat het via een wiskundige formule werkt die software en gebruik aan elkaar koppelt. Je krijgt dus een aanbeveling omdat je iets anders al gebruikt.

\textbf{Bas:} Volgens mij komt dat ook een beetje omdat je op wakoopa niet direct op zoek bent naar iets nieuws. Bij bol.com ben je echt op zoek naar een boek. Bij wakoopa heb je al je programma's, je zit in een bepaalde comfort zone. Je weet bijvoorbeeld hoe photoshop 6 werkt, en een aanbeveling zou dan photoshop 7 kunnen zijn. Je geeft dan aan dat ze uit hun comfort zone moeten, en even tijd steken in het leren van iets nieuws, en veel mensen hebben dan zoiet van "wacht even". Een van de gouden regels van een veranderingsproces is dat mensen pas veranderen als ze daar zelf inbreng in hebben. Software is een keuze, maar je kan ook denken "dit is de aanbeveling en dit is waarom", en dan kan je er bij bedenken dat ze iets er zelf ook over kunnen beslissen. Dat proces is dan veel gemakkelijker.

\textbf{Kilian:} Eigelijk zou je dus een aanbevelingstraject zou moeten hebben. Nu geven we een lijst, maar misschien kan je laten zien wat andere mensen beter vinden aan dat nieuwe programma.

\textbf{Bas:} Stel dat je photoshop gebruikt, en als aanbeveling illustrator hebt. Dat je dan aan kan geven dat het handig is voor logodesign. En dat je dan kan zeggen "ik maak geen logo's, ik maak websites" en dat je niks met vectoren doet. Dat er dan uitkomt dat je illustrator niet moet gaan gebruiken, of juist dat je wel logo's maakt en dat illustrator dan wel een goede tips is. Je moet aangeven wat het doel is voor een gebruiker.

\textbf{Kilian:} in de afgelopen paar jaar zijn er usability methodes bijgekomen, zoals video's die veel effectiever blijken, of google suggest die je suggesties geven. Zijn er dingen die er voor jou uitspringen op dat gebied?

\textbf{Bas:} Nou, wat ik vaak tegenkom en gebruik zijn wel de video's, dat geeft gelijk een beeld. Google suggest gaat volgens mij nog niet ver genoeg, ik wil graag dat als ik fiets zoek dat er dan ook "probeer eens rijwiel" bij komt. Ik ben niet heel veel bezig met dat soort dingen. Het is meer op gevoel, dat het hele plaatje klopt.

\textbf{Kilian:} Hoe onderzoek je zelf sites op usability? Waar let je dan precies op?

\textbf{Bas:} Waar ik op let is pure acties. Ik maak vooral commerciele sites. Op elke pagina moet er een actie voor een gebruiker zijn. Een voorbeeld wat ik laatst in een website heb gedaan is het toevoegen van een offerteformulier, naast een contactformulier. Je merkt dat mensen dat toch gaan gebruiken en dat er een week later al offerteaanvragen binnen komen. Volgens mij is dat het belangrijkste, dat er altijd een actie voor de gebruiker is.

\textbf{Kilian:} Welke methodieken gebruik je zelf voor usability onderzoek?

\textbf{Bas:} Ik doe eigelijk enkel expert reviews en statistiekanalyse. Uiteindelijk moet een klant er voor betalen en gebruikersonderzoek neemt veel tijd in beslag. Ik wil het wel doen, maar het geld daarvoor is er niet bij de klanten die ik nu heb. Ik kijk naar de paden die mensen op een site doorlopen, maar dat heeft natuurlijk beperkingen. Je doet aannames op basis van de gegevens die je hebt. Je weet bijvoorbeeld niet of mensen boven of onderaan op een knop klikken, enkel naar welke pagina's te gaan. Ook weet je vaak niet met wat voor doel ze op een site komen, wat voor soort mensen het zijn.

\textbf{Kilian:} Waar zie jij het vakgebied in de toekomst heen gaan?

\textbf{Bas:} Er zijn nu heel veel verschillende online tools en dingen die bepaalde dingen doen, zoals bijvoorbeeld Silverback. Google die doet ook steeds meer, analytics heeft nu bijvoorbeeld een api waardoor je veel meer kan analyseren, hoewel je je wel af kan vragen of sommige van die dingen wel geanalyseerd moeten worden. Volgens mij komen er in de toekomst veel meer tools die al dat soort dingen samen doen, als groot omvattend iets zegmaar. Door met meer data verbanden te leggen kan je weer betere aannames maken. Het is nu heel erg versplinterd, en ik denk dat dat langzaam samen zal komen.

