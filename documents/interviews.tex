\chapter{Interviews}
\label{interviewappendix}
Er zijn drie usability experts geinterviewd over hun ervaringen met learning networks en usability en hun visie daarop. Voor dit interview is een losse structuur aangehouden, met de volgende vragen als leidraad:

\begin{itemize}
  \item Welke learning networks gebruik je zelf?
  \item Welke learning networks zijn goede voorbeelden van usability en welke minder?

  \item Welke struikelblokken kom je tegen op learning networks's tijdens onderzoeken
  \item Hoe ga je daar mee om tijdens het beoordelen of ontwerpen van een learning networks, wat zijn oplossingen?

  \item Zijn er recente stijl/designontwikkelijken, op usability gebied (zoals bijvoorbeeld google suggest) die je goed vind
  \item En slecht?

  \item Welke onderzoeksmethodieken zijn jouw favoriet? Als je veel budget hebt, en als je weinig budget hebt?
  \item Is er een methode die je verrast heeft toen je hem voor het eerst gebruikte?

  \item Welke methodieken vind jij handig voor learning networks?
  \item Welke vind je minder handig?

  \item Waar zie jij het vakgebied in de toekomst heengaan, wat zijn opkomende technieken volgens jou?
\end{itemize}

Er is gekozen voor deze losse structuur omdat usability een heel breed gebied is. Iedere expert heeft een specialisatie en door de vrijheid te laten kan de expert zijn expertise toepassen op de gestelde vragen.

\section{Bas Bakker}
Bas Bakker is een zelfstandig ondernemer die is afgestudeerd op het ontwerpen en bouwen van een learning network, `goedkado.nl'. Dit is een website waarop mensen kadotips voor vrienden kunnen krijgen aan de hand van integratie met bijvoorbeeld hyves. Bas maakt geregeld expert reviews voor bestaande websites.

\paragraph{}
\textbf{Kilian:} Welke learning networks ben je zelf op actief?

\textbf{Bas:} Ik heb net geleerd dat hyves geen learning network is. Ik gebruik linkedin, flickr, en youtube gebruik ik wel eens, maar wie doet dat nou niet? Twitter gebruik ik ook, maar ik geloof niet dat dat er onder valt. Ning, ik weet niet of dat er onder valt, dat ligt eraan welk ning netwerk natuurlijk. Dit waren zowiezo wel de hoofddingen die ik gebruik.

\textbf{Kilian:} Heb je misschien voorbeelden van hoe Ning ingedeeld kan worden?

\textbf{Bas:} Er zijn twee verschillende voorbeelden. De een is vooral netwerken en de andere is inderdaad meer community waar je ook leert. Wat je veel ziet is dat nings worden opgericht op basis van een offline netwerk. mensen die een community maken om in contact met elkaar te blijven. Opencoffee leiden is daar een voorbeeld van, maar daar gebeurt niet zoveel. Een ander voorbeeld is bijvoorbeeld die van de zaanstreek, daar laten ze zien hoe mooi de zaanstreek is, plaatsen er foto's op en er zijn discussies. De dingen waar ik Ning het meest voor gebruik beginnen offline en gaan dan ook online.

\textbf{Kilian:} Als je kijkt naar ning, of learning networks in het algemeen, zijn er dan gelijk al dingen die je zelf anders zou doen op het gebied van usability?

\textbf{Bas:} Daar moet ik even over denken. Ik vind het concept van ning best slim, zakelijk, en voor niks kan je een netwerk opzetten. Qua usability laat het te wensen over. Het gaat nu vooral over "nu en wat er nog komen gaat". Een gebeurtenis verdwijnt uiteindelijk en dan moet je gaan graven om dat terug te vinden. Dus om een community in te gaan is vrij moeilijk. Dat zou beter kunnen. En de algemene usability...Je moet het een keertje doen, en dan ga je het pas begrijpen. Dat zie je bij heel veel communities, Je komt er, en dan is er geen doel voor jou. Het is vrijheid blijheid, en je kan alle kanten op. Mensen verliezen zich daar in. Dat zie je bij twitter. Mensen zitten daar zonder dat ze weten wat ze aan het doen zijn. Daarnaast draagt het ook niet altijd bij aan het netwerk.

Een voorbeeld daarvan is bij meetup, een website om bij conferenties te zeggen dat je ook komt, de grootste groep daar is werkeloze huismoeders die thuis zitten en met elkaar in contact willen komen. Ieder sociaal netwerk zorgt ervoor dat mensen zelf hun doel kunnen bepalen. Als ze dat hebben gevonden gaan mensen er wel heen, alleen wanneer ze beginnen denken veel mensen "wat moet ik hier mee?". Bij ning zie je dat als je er komt, dan weet je niet wat er in het verleden is gebeurd, en dan weet je niet wat je daar mee moet. Wat je ziet is dat mensen zich aanmelden en ze worden actief, of ze komen nooit meer terug.

\textbf{Kilian:} Ning is dus goed als ondersteunend van een offline community, dus wat je meer moet hebben is dat mensen in het verleden dingen hebben gedaan en daar over verder praten en dingen mee kunnen doen in plaats van enkel in de toekomst kijken.

\textbf{Bas:} Op zich zijn er ook wel genoeg netwerken op Ning die enkel online zijn. Dan nog is het heel moeilijk om er in te graven. Het is een forum, en als je eenmaal 100 discussies verder bent weet je niet wat er heeft gespeeld. Als je er al een tijdje bent is dat geen probleem, maar als nieuwe gebruiker moet je eerst een hoop onderzoek doen naar wat er al is gebeurd. Als je dat niet doet wordt je direct raar aangekijken. Ning is eigelijk net een stap verder dan een forum want het kan iets meer, maar het is slechts éen stapje verder. Het mist het directe doel.

\textbf{Kilian:} Is dat iets wat jij als je zelf iets zou bouwen, wat jij zou doen? Als je op de site komt dat je gelijk aangeeft "dit is het doel"?

\textbf{Bas:} Dat ligt eraan wat het doel is van wat je bouwt. Bij een klant die een sociaal netwerk wilt maken zoals Hyves, dan zou ik dat niet doen. Maar als je een netwerk rondom initiatieven maakt, dan moet je wel neerzetten wat het initiatief is. Bijvoorbeeld bij Flickr is het heel duidelijk dat het om foto's gaat. Als het echt om éen ding gaat, als learning network, dan moet je dat wel neerzetten ja. Bij hyves kan je het niet in drie zinnen neerzetten omdat er zoveel manieren zijn.

\textbf{Kilian:} Hoe zou dat doel gepresenteerd moeten worden? Als die drie regels uitleg, of een complete tutorial?

\textbf{Bas:} Ja, ik zou het allebij doen. Het moet zowiezo in éen oogopslag duidelijk zijn. Voor de "luie bezoeker" moet het gelijk duidelijk zijn, en als iemand meer informatie wilt, dan moet het ook beschikbaar zijn, dan moet er ook die tutorial zijn. Maar als die eerste doelen weg zijn dan moeten mensen dus een tutorial doorlopen om te weten waar het over gaat. Bijvoorbeeld Google wave, dat kwam uit en toen moest je een film van een uur kijken. Ik heb het niet gedaan. Toen ging ik google wave kijken en ik snapte er niks van. Nu heb ik er een beetje mee gespeeld en is het wel leuk, maar dat is een beetje dat vrijheid blijheid. Er zit geen doel achter, en dan moet je veel uitleggen. Maar ik denk op het moment dat je zegt "dit zijn de doelen" dan is dat helder, en daarna met meer informatie als mensen dat willen.

\textbf{Kilian:} Een recente trend om een learning network in een video uit te leggen. Wat vind jij daar van?

\textbf{Bas:} Als je kijkt hoe dat werkt met leren, dan leer je eerst iets, dat commit je en dan ga je het doen, en dan ga je weer leren, commiten, doen en zo voorts. Dus je zal het eerst moeten leren. Qua design kan je dingen zo duidelijk mogelijk maken. Bijvoorbeeld met een duidelijke call to action. Ook hier is beide weer beter. Een video maken kost gewoon tijd, maar het is wel duidelijk. Mensen focussen zich op een video, en op een site niet. Als je een video aanzet, zijn mensen daarop gefocussed terwijl ze op een site overal kijken. Alleen mensen verplichten is weer geen goed idee. Je moet een balans vinden tussen een intuitief design, en daarnaast een video om het uit te leggen. Je hebt altijd de helft blijft en de helft gaat wel weg.

\textbf{Kilian:} Dus zoveel mogelijk verschillende manieren werkt volgens jou het best?

\textbf{Bas:} nouja, niet zoveel mogelijk, maar wel meerdere.

\textbf{Kilian:} De site van jouw afstudeerproject, goedkado, is ook een soort learning network, ben je daar ook problemen tegen gekomen qua usability?

\textbf{Bas:} Ja, daar moet ik even over denken. Een van de dingen was dat de navigatie aan de linkerkant stond en totaal over het hoofd werd gezien. Ik had een aanname gedaan door dit over te nemen van bol.com en amazon die dat ook allebei doen. Mensen zagen dat niet. Dat was gemakkelijk op te lossen door de navigatie bovenaan overdwars te zetten. Een ander punt is dat de labels van navigatie en de structuur van de site. Je doet aannames over wat je doet en koppelt daar namen aan. Bijvoorbeeld 'kadotips'. We geven je kadotips voor vrienden, maar het is heel onduidelijk wat dat betekend omdat mensen daar niet bekend mee zijn. Ze weten niet voor wie, en wat daar achter zit. Ze wilden zoeken bij kado's, en niet bij kadotips. Jouw gedachten en aannames sluiten soms niet aan bij wat de meeste gebruikers vinden. Er zit ook een deel uitleggen bij, mensen wisten niet waarvoor die kadotips dan waren, en hoe ze gemaakt werden. Ze willen weten hoe het werkt. Die beschrijvingen zou er ook bij moeten.

\textbf{Kilian:} Het gaat dus erg om de uitleg, en mensen willen het begrijpen en daar moet je in voorzien.

\textbf{Bas:} Ja dat klopt. Je probeert een probleem op te lossen, maar in de loop van de jaren zijn mensen daar wantrouwig naar geworden. Ze geloven niet dat dat vanzelf gaat, dus ze willen weten hoe het probleem wordt opgelost. Niet technisch natuurlijk, maar gewoon in normale taal, dat inzicht.

\textbf{Kilian:} Je zegt dus eigelijk dat in plaats van aangeven dat er een probleem is, wat vaak gebeurt, dat je ook moet uitleggen hoe je dat op kan lossen?

\textbf{Bas:} Dus eigelijk, dat je het probleem oplost is niet genoeg, je moet het oplossen. Het heeft vooral met vertrouwen te maken. Dit zie je ook bij de aanbevelingen op bol.com en amazon. Ik bekijk dat altijd, maar het is wel interessant om te weten hoe die gemaakt worden. Misschien past dat wel helemaal niet bij mij. Ik kan zelf wel een selectie van interessante aanbevelingen maken, maar het is ook interessant om te weten wat voor soort mensen dan precies die aanbevelingen kochten. Met dat kan ik gelijk een selectie maken van wat wel en niet interessant is voor mij. Het komt dus terug op dat je het vertrouwen moet krijgen. Je moet kijken naar de twee soort personen. Je hebt mensen die snel keuzes maken, en mensen die meer de diepte in willen en snel kunnen schakelen.

\textbf{Kilian:} Dat klopt ook wel met een artikel wat ik voor mij scriptie hebt gebruikt over persona's, daar zijn vier typen mensen. Er zijn mensen die snel gebaseerd op feiten of snel gebaseerd op emotie en mensen die sloom gebaseerd op feiten of snel gebaseerd op emotie beslissen. Dat klopt ook wel met jouw indeling volgens mij.

\textbf{Bas:} Ik denk inderdaad dat het zo werkt. Iedereen is natuurlijk uniek in zijn denken. Iedereen heeft zijn eigen perspectief. Maar op het moment dat jij een van de twee groepen niet faciliteerd, dan verlies je ze. Specialisten die willen de feiten en weten hoe ze dat kunnen controleren, en de generalisten willen juist snel kunnen zien waar het over gaat. Daar moet je een balans tussen vinden.

\textbf{Kilian:} Op Wakoopa staan ook recommendations, en daar is ook veel behoefte aan uitleg. Volgens mij komt dat omdat er vooral de specialisten op wakoopa zitten, en die willen volgens mij vaak weten hoe het precies zit. Dat is lastig omdat het via een wiskundige formule werkt die software en gebruik aan elkaar koppelt. Je krijgt dus een aanbeveling omdat je iets anders al gebruikt.

\textbf{Bas:} Volgens mij komt dat ook een beetje omdat je op wakoopa niet direct op zoek bent naar iets nieuws. Bij bol.com ben je echt op zoek naar een boek. Bij wakoopa heb je al je programma's, je zit in een bepaalde comfort zone. Je weet bijvoorbeeld hoe photoshop 6 werkt, en een aanbeveling zou dan photoshop 7 kunnen zijn. Je geeft dan aan dat ze uit hun comfort zone moeten, en even tijd steken in het leren van iets nieuws, en veel mensen hebben dan zoiet van "wacht even". Een van de gouden regels van een veranderingsproces is dat mensen pas veranderen als ze daar zelf inbreng in hebben. Software is een keuze, maar je kan ook denken "dit is de aanbeveling en dit is waarom", en dan kan je er bij bedenken dat ze iets er zelf ook over kunnen beslissen. Dat proces is dan veel gemakkelijker.

\textbf{Kilian:} Eigelijk zou je dus een aanbevelingstraject zou moeten hebben. Nu geven we een lijst, maar misschien kan je laten zien wat andere mensen beter vinden aan dat nieuwe programma.

\textbf{Bas:} Stel dat je photoshop gebruikt, en als aanbeveling illustrator hebt. Dat je dan aan kan geven dat het handig is voor logodesign. En dat je dan kan zeggen "ik maak geen logo's, ik maak websites" en dat je niks met vectoren doet. Dat er dan uitkomt dat je illustrator niet moet gaan gebruiken, of juist dat je wel logo's maakt en dat illustrator dan wel een goede tips is. Je moet aangeven wat het doel is voor een gebruiker.

\textbf{Kilian:} in de afgelopen paar jaar zijn er usability methodes bijgekomen, zoals video's die veel effectiever blijken, of google suggest die je suggesties geven. Zijn er dingen die er voor jou uitspringen op dat gebied?

\textbf{Bas:} Nou, wat ik vaak tegenkom en gebruik zijn wel de video's, dat geeft gelijk een beeld. Google suggest gaat volgens mij nog niet ver genoeg, ik wil graag dat als ik fiets zoek dat er dan ook "probeer eens rijwiel" bij komt. Ik ben niet heel veel bezig met dat soort dingen. Het is meer op gevoel, dat het hele plaatje klopt.

\textbf{Kilian:} Hoe onderzoek je zelf sites op usability? Waar let je dan precies op?

\textbf{Bas:} Waar ik op let is pure acties. Ik maak vooral commerciele sites. Op elke pagina moet er een actie voor een gebruiker zijn. Een voorbeeld wat ik laatst in een website heb gedaan is het toevoegen van een offerteformulier, naast een contactformulier. Je merkt dat mensen dat toch gaan gebruiken en dat er een week later al offerteaanvragen binnen komen. Volgens mij is dat het belangrijkste, dat er altijd een actie voor de gebruiker is.

\textbf{Kilian:} Welke methodieken gebruik je zelf voor usability onderzoek?

\textbf{Bas:} Ik doe eigelijk enkel expert reviews en statistiekanalyse. Uiteindelijk moet een klant er voor betalen en gebruikersonderzoek neemt veel tijd in beslag. Ik wil het wel doen, maar het geld daarvoor is er niet bij de klanten die ik nu heb. Ik kijk naar de paden die mensen op een site doorlopen, maar dat heeft natuurlijk beperkingen. Je doet aannames op basis van de gegevens die je hebt. Je weet bijvoorbeeld niet of mensen boven of onderaan op een knop klikken, enkel naar welke pagina's te gaan. Ook weet je vaak niet met wat voor doel ze op een site komen, wat voor soort mensen het zijn.

\textbf{Kilian:} Waar zie jij het vakgebied in de toekomst heen gaan?

\textbf{Bas:} Er zijn nu heel veel verschillende online tools en dingen die bepaalde dingen doen, zoals bijvoorbeeld Silverback. Google die doet ook steeds meer, analytics heeft nu bijvoorbeeld een api waardoor je veel meer kan analyseren, hoewel je je wel af kan vragen of sommige van die dingen wel geanalyseerd moeten worden. Volgens mij komen er in de toekomst veel meer tools die al dat soort dingen samen doen, als groot omvattend iets zegmaar. Door met meer data verbanden te leggen kan je weer betere aannames maken. Het is nu heel erg versplinterd, en ik denk dat dat langzaam samen zal komen.

\section{Ruben Timmerman}
Ruben Timmerman Is eigenaar van Usearchy en doet veel expert reviews en gebruikersonderzoek voor andere bedrijven. Zijn meest recente project is het aan de wieg staan van het Hyves herontwerp en het begeleiden van dat proces.

\paragraph{}
\textbf{Kilian:} Welke learning networks gebruik je zelf?

\textbf{Ruben:} Flickr en Wakoopa, linkedin, maar dat is misschien meer sociaal. Waarschijnlijk nog veel meer, maar dan zou ik even moeten denken. Heb je nog suggesties?

\textbf{Kilian:} Bijvoorbeeld iets als linksharing

\textbf{Ruben:} Doe ik weinig aan, af en toe slideshare. Delicious gebruik ik eigelijk bijna niet meer. Ik gebruik google docs, maar ik weet niet of dat een learning network is. Je shared je docs, maar het is natuurlijk niet open. Je kiest je eigen network. Wij gebruiken hier enkel google docs, maar het is wat minder vrij

\textbf{Kilian:} Welke learning networks hebben hun usability goed op orde, en welke niet?

\textbf{Ruben:} Eigelijk allemaal wel. Ik ben een late adopter, dus de networks die ik gebruik, gebruik ik omdat ze al succesvol zijn. Ik heb nu google wave bijvoorbeeld nog niet gebruikt omdat ik het nut er nog niet van in zie. Dan ga ik het pas gebruiken. Er zijn niet echt networks waar ik ontevreden over ben. Van slideshare kan ik soms wel zeggen dat het niet helemaal geweldig is. Van Flickr heb ik soms echt wel van "wow, wat hebben ze dat slim of handig gedaan". Maar slideshare kan ik prima gebruiken, dus ik zie daar niet echt problemen mee.

\textbf{Kilian:} Heb je al naar google wave gekeken?

\textbf{Ruben:} Nog niet echt. Wat ik hoop dat gaat gebeuren is dat andere bedrijven het googlewave protocol gaan gebruiken voor hun aplicaties, en dat die usability er dan dus al ingebakken zit.

\textbf{Kilian:} Jij doet ook wel eens expert reviews op learning networks. Wat zijn daar de struikelblokken?

\textbf{Ruben:} Ik kijk dan toch stiekem even naar hyves, en dan de delen die onder een learning network kunnen vallen. De herbruikbaarheid van objecten binnen verschillende concepten is een enorm probleem daar. Wat we bijvoorbeeld hadden was het object foto, daar moet je vanalles mee moeten kunnen doen. Dat kan in verschillende contexten, zoals bijvoorbeeld in een bericht, of in een album. Je hebt dan hetzelfde concept ("het bewerken van een foto") in een andere context. We hebben alle objecten en contexten uitgeschreven, dat zijn honderden pagina's waarbij we voor alle objecten opgeschreven waar en op welke manier ze gebruikt worden. Het grootste probleem of uitdaging is volgens mij zorgen dat dat over de hele site klopt.

\textbf{Kilian:} Bij wakoopa hebben we heel erg dat het benoemen van delen of functies. Hoe jij er over denkt is heel anders dan hoe anderen daar over denken.

\textbf{Ruben:} Bij hyves hebben we dat ook altijd gehad. Hoe zij dat oplosten was door middel van een courtesylinkje, en elke keer was dat zo. Gebruikers verwachten dingen op andere plekken, en op een gegeven moment ben je overal linkjes aan het plaatsen. Dat is een probleem

\textbf{Kilian:} Overal linkjes is natuurlijk ook niet optimaal, hoe ga je daar mee om?

\textbf{Ruben:} Waar je dan op komt is dat je je gebruikers eigelijk wilt educaten. Je wilt ze iets gaan leren. Je moet iets bedenken waarbij je de link tussen objecten, dus niet een echte link, maar de link in iemands hoofd, duidelijker moet gaan maken. Door gebruik van hetzelfde icoon, of eenduidig taalgebruik of iets dergelijks. Waar we bij hyves heel vaak op kwamen was dat een aantal verschillende elementen die in de loop van de jaren zijn gemaakt eigelijk heel erg op elkaar lijken. Wat we dan moeten doen is ze groeperen of op elkaar te laten lijken zodat de band tussen de elementen duidelijk is. Je wilt dat omgooien en hetzelfde maken, maar dat moet je dan stapje voor stapje doen. Door dat stukje bij beetje online te zetten zorg je dat gebruikers langzaam leren wat de relatie tussen objecten is. Dat is moeilijk want je maakt toch een beslissing om alles om te gooien.

\textbf{Kilian:} Maak jij veel gebruik van design patterns, kijkend naar het samenvoegen van die concepten?

\textbf{Ruben:} Ja, naast dat je objecten in meerdere contexten kan hebben, zoals een foto editten in een bericht of in een fotoalbum, krijg je ook dat je een video kan editten, en een krabbel kan editten. En daar wil je ook regels voor. Bij hyves is het voornamelijkste wat we daar gedaan hebben is het tekstveld. de Rich text editors verschilde heel erg, en daar hebben we een ding van gemaakt. Een van de eerste dingen die ik daar heb geroepen is dat ze patterns moesten gaan maken. De patterns en de context zijn dus eigelijk de twee dimensies.

\textbf{Kilian:} Zijn er recente stijlontwikkelingen die jou opvallen op het gebied van usability? zoals google suggest, of de google wave scrollbars, of iets heel anders als een video op de homepagina.

\textbf{Ruben:} Moeilijk om daar snel antwoord op te geven. De truuk is om het wiel niet opnieuw uit te vinden. Als jij een learning network hebt, begin met de succesvolste en bouw die naar. Met Eduhub heb ik hetzelfde gedaan, dat is een vergelijkingssite voor opleidingen, en daar heb ik booking.com en funda voor nagebouwd. Pas daarna ben ik zelf dingen aan gaan passen. Maar om trends te noemen. Je ziet ze op allerlei niveau's. In vergelijkingsland is het bijvoorbeeld normaal om meerdere resultaten te vergelijken door middel van vinkjes naast de resultaten. In de e-commerce zie je ze nog sterker, zoals bijvoorbeeld direct onder de bestelbutton een goed-gevoel zinnetje te zetten. Iets als "het is op vooraad, en u heeft het altijd binnen een dag in huis". Dat soort patronen zie je heel erg terug. Je ziet ook dat de grote namen elkaar nadoen of tot dezelfde oplossing komen. Als de interaction designers goed nadenken komen ze vaak tot bijna dezelfde conclusie, omdat ze allen hetzelfde probleem hebben en goed naar hun gebruikers luisteren.

\textbf{Kilian:} Wat zijn jouw favoriete onderzoeksmethodieken?

\textbf{Ruben:} Ik heb heel veel kwalitatieve onderzoeken gedaan, bij mensen thuis of op kantoor of in een lab, en dat is voor opdrachtgevers het leukst, want die krijgen dan coole videos. Veel opdrachtgevers vragen daar om, maar wat er op een gegeven moment gebeurde was twee dingen. Je hebt sites die nog nooit getest zijn en waar een niet zo heel goede ontwerper aan heeft gewerkt, en die zijn eigelijk zo slecht dat je beter een expert review kan doen. Daar kan ik zo al de grote problemen voor aangeven, en dan kan je het geld wat overblijft gebruiken voor een goede ontwerper. Je hebt maar zoveel geld om het te verbeteren, en dan kan je beter een aantal problemen vinden en die ook oplossen, dan een heleboel problemen. Wat daar wel kan is dat je na de eerste verbeteringsronde een gebruikersonderzoek doet. De andere groep sites is eigelijk te goed voor gebruikersonderzoek. Als je kijkt naar booking.com, dat is echt een goede site. Die is jarenlang doorgetest. Je kan daar wel gebruikersonderzoek doen, maar dan is het eigelijk enkel handig als inspiratie voor a/b-tests. Je kan dan inspiratie uit nog meer bronnen halen, zoals interviews, of vragen aan medewerkers of misschien een enquetetool. Daarmee kan je dan met echte tests aan de slag. Bij dat soort goede sites zeg ik tegenwoordig dat ze geen onderzoek moeten doen om problemen te vinden, maar gewoon gelijk te testen op de site zelf.

Wat belangrijk is, is het proces van feedback. Je moet continue feedback hebben en daar mee omgaan. Dan heb je altijd feedback om nieuwe testen op te baseren. Dan kan je heel snel kijken wat het beste werkt. Voor mij zijn de a/b tests en ingebouwde continueue tests, die kijken `welke invloed heeft dit op mijn bankrekening', dus waar het echt om gaat, de heilige graal.

\textbf{Kilian:} Dus je vind, eigelijk moet je niet zozeer onderzoeken doen, maar meer testen

\textbf{Ruben:} Ja, je moet de methodieken gebruiken om feedback en inspiratie te krijgen, en daarmee kwantitatieve tests doen. Dat is mijn filosofie.

\textbf{Kilian:} Kan je dan zeggen dat het veld van usability zo volwassen is geworden dat usability experts zelf al kunnen zeggen wat de echte problemen zijn?

\textbf{Ruben:} Dat niet, maar wel om op een bepaald niveau te komen. Tot dat niveau weet een usability expert meer dan genoeg om je bezig te houden. Voor slechte sites ben ik het met je eens, maar voor sites die een beetje op niveau zit geldt dat niet meer. Er zijn dan zoveel dingen die kunnen, dat je ze gewoon moet testen. Het proces op orde krijgen is denk ik het belangrijkst. De echte usability experts zijn straks niet meer de mensen die zeggen hoe het moet, maar die zeggen hoe je je proces anders moet inrichten. Dat wordt hip de komende jaren.

\textbf{Kilian:} Zijn er methodieken die jij weinig toe vind voegen?

\textbf{Ruben:} Eentje die ik wel een beetje overrated vind zijn online enqu\^etes. Die worden te pas en te onpas op sites worden geknalt. Volgens mij worden die vooral gebruikt om dingen goed te praten. De NOS heeft dat nu bijvoorbeeld, waarin ze letterlijk vragen `welk cijfer geeft u het design van deze site?'. Dan gaan ze na het herontwerp nog eens tienduizend mensen lastig vallen, en dan hopen ze dat er eerst een 6.9 en nu een 7.1 uitkomt, en dat ze dan tegen de manager kunnen zeggen `kijk, het is een verbetering!'. Online enqu\^etes worden volgens mij misbruikt. De bezwaren zijn al lang bekent, maar iedereen gebruikt ze omdat ze zo gemakkelijk lijken. Je krijgt feedback, het is te quantificeren en je kan het aan je manager laten zien. Alleen, enkel bepaalde mensen doen er aan mee, je irriteerd er een hoop gebruikers mee. Er zitten zoveel nadelen aan. Ze hebben absoluut hun plek in het proces, maar enkel als een klein onderdeel.

\textbf{Kilian:} Kom dat misschien omdat die enqu\^etetools altijd direct zichtbaar worden, als je de site nog niet hebt gebruikt?

\textbf{Ruben:} Ja precies. De betere enqu\^etetools zeggen nu `mogen we uw gebruik monitoren en als u van de site afgaat een enqu\^ete laten invullen?'. Maar als je weggaat heb je al helemaal geen zin meer in vragen. Dus daar is vanalles mis mee. Maar omdat het naar managers gemakkelijk is, wordt het dus gebruikt. Het proces wordt gemakkelijker, maar het is niet goed.

\textbf{Kilian:} Welke methodieken vind je goed om voor learning networks te gebruiken?

\textbf{Ruben:} Voor learning networks is het nog meer zo dat het feedbackproces geintegreerd moet zijn, omdat ze zo dicht bij hun gebruikers staan. Hyves heeft bijvoorbeeld een `leuker kunnen we het niet maken, wel gemakkelijker'-hyve, en wakoopa heeft getsatisfaction. Voor elk bedrijf is dat anders, maar feedback moet dus heel gemakkelijk zijn.

\textbf{Kilian:} Hoe vind je uit wat voor jou het beste werkt?

\textbf{Ruben:} Daar kan je dan weer goed een expert voor gebruiken. Die kan dan bijvoorbeeld zeggen dat getsatisfaction te nerdy is voor de doelgroep. Je moet meerdere dingen proberen. Dat proces moet je als bedrijf aandurven. Je organisatie moet er op ingestelt zijn om meerdere dingen te proberen.

\section{Bojhan Somers}
Bojhan Somers is een user experience developer. Hij is user experience lead van het open source project Drupal en heeft daar de leiding over de developers en designers en voert daar veel gebruikerstesten voor uit.

\paragraph{}
\textbf{Kilian:} Welke learning networks gebruik jij zelf?

\textbf{Bojhan:} Delicious gebruik ik vrij intensief, en flickr gebruik ik ook wel. Dat zijn ze denk ik wel. Je hebt er meestal maar een aantal waar je intensief mee bezig bent. Vimeo gebruik ik ook nog wel, maar meer voor sharen. Slideshare gebruik ik heel soms.

\textbf{Kilian:} Wat vind je in het algemeen van de usability van die learning networks?

\textbf{Bojhan:} Flickr in zijn begintijd was niet het toppunt van usability. Vooral het uploaden was niet goed. het was moeilijk om dingen te taggen en bij te houden. Het is bij veel tools dat als er veel informatie op het scherm staat dan ziet het er goed uit, want daar zijn ze op ingesteld, maar bijvoorbeeld bij vimeo gaan ze uit van een kleiner aantal dingen, zeg tien, en ik heb er veertig, dan houd de interface je echt tegen om je content goed te managen. De opties die er zijn zijn verwarrend qua interface.

Bij flickr en delicious is voor mij het voordeel dat ik tools gebruik die op mijn pc staan, dus eigelijk weinig met de web interface te maken heb.

\textbf{Kilian:} Welke struikelblokken kom je tegen op learning networks?

\textbf{Bojhan:} Bij delicious gaat het er bij hun data heel erg rich is, veel tags en descriptions enzo, en meestal doe je dat niet. Dan heb ik honderd links en die moet ik dan gaan opschonen. Dan heb ik zoiets van `moet ik dat nou gaan opschonen, hebben jullie daar niet iets van intelligentie voor?'. Dat is vooral iets, je verwacht meer intelligentie omdat zij die data al hebben.

\textbf{Kilian:} Sommige dingen wil je je als gebruiker dus niet mee bezig houden?

\textbf{Bojhan:} Zeker, en je ziet bijvoorbeeld picasa van google, die heeft heel veel intelligentie, zoals gezichtsherkenning. Dat is is wel heel mooi, bijvoorbeeld bij familiefoto's is dat heem gemakkelijk. Ik denk inderdaad dat je echt verwachtingen hebt over de intelligentie van een systeem, dan zitten de roadblocks er wanneer een systeem niet aan jouw verwachtingen voldoet.

\textbf{Kilian:} Jij bent zelf veel met de user experience van drupal bezig, kan je daar wat over vertellen?

\textbf{Bojhan:} Ik ben user experience lead in het project, dus ik stuur eigelijk andere designers aan, en waar ik heel erg mee bezig ben is het grotere idee. Niet per se pagina per pagina, maar meer de workflow van de applicatie, en de samenhang van componenten. Wat bij drupal heel erg het geval, en je hebt modules, en die maken drupal. Het is heel normaal om dertig modules te hebben en die samen te gebruiken. Dus er moet een soort van natuurlijke samenhang zijn. Het is dus aan de core om te zorgen dat de patterns et cetera consistent zijn.

Het is heel veel samenwerken met ontwikkelaars en kijken hoe hun perceptie is van wat bruikbaar is, en soms hoe we die kunnen doorbreken. Er worden soms echt wel foute keuzes gemaakt. You can't blame them, maar het is wel zo. Ik leer heel veel over in hoe je goed kan communiceren binnen een community, over de redenen achter een design. Het is heel erg balanceren tussen de usability en bijblijven bij de rest.

\textbf{Kilian:} Hoe communiceer jij met de rest van de community?

\textbf{Bojhan:} Alles gaat via een issue queue proces, en daarnaast IRC en een mailinglist, maar die mailinglist was niet accessible genoeg, dus daar zijn we van af aan het stappen. De issue queue zit geintegreerd in alles wat we doen. Het werkt echt heel goed. Het enige nadeel is dat je geen goede discussies kan hebben. Het is niet threaded en daardoor komen er al snel veel posts. Je kan niet het hele concept er neerzetten, je moet het in hele kleine stukjes opbreken zodat vrijwilligers er ook wat mee kunnen.

Je hebt maar een paar uur in de week om aan dat open source project te werken, en dan moet het duidelijk zijn wat je kan doen en hoe je dat kan doen. Wat cool is is dat je met allerlij experts op een hoop gebieden samenwerkt.

\textbf{Kilian:} Hebben jullie een soort van core design pattern library?

\textbf{Bojhan:} We zitten nu nog heel erg in een verandering van "code eerst" naar "design eerst". Dat moet eerst gedaan zijn voordat we maar aan dat soort dingen kunnen denken. We zijn meer met de informatie architectuur bezig.

Het is moeilijk om mensen te vinden voor de user experience. Het is een complexe applicatie, en niet heel veel designers hebben daar ervaring mee. en dan is het ook nog eens open source, dus enkel vrijwilligers.

\textbf{Kilian:} Jullie zijn dus de informatiearchitectuur van drupal aan het omgooien, hoe ver gaat dat?

\textbf{Bojhan:} Alles raakt het. De Drupal versie van 5 jaar geleden heeft het ontwerp helemaal ongegooid, heel netjes met onderzoek, maar bij de versies daarna merkten we dat het aantal modules wat je gebruikt gewoon heel erg verhoogd is, en daardoor konden mensen niks meer vinden. Ze doen nu gewoon ctrl+f op de pagina om functionaliteit te vinden.

We zijn meer naar het mentale model van een gewoon CMS gegaan, met secties en pagina's, terwijl we eerst eigelijk slechts een pagina hadden.

De developers die aan core werken zijn echt de powerusers, dus je moet hun overtuigen dat het misschien voor hun geen probleem is, maar voor de andere 90% van de gebruikers wel. Dat is ook een probleem.

\textbf{Kilian:} Hoe overtuig je die developers?

\textbf{Bojhan:} Usability testing. Echt laten zien dat gebruikers hier echt problemen mee hebben. We hebben best veel data, we hebben zestig gebruikerstests gedaan in het afgelopen anderhalf jaar, dus we kunnen echt aangeven wat de echte problemen zijn.

We houden dan presentaties op conferenties, en heel erg uitleggen waar de gebruikers fout gaan. En dus een beetje pushen, want je veranderd werkwijzen.

\textbf{Kilian:} Als je met mensen samenwerkt wordt het altijd een beetje verandermanagement, zei Bas die ik gister interviewde. Die vertelde dat als je dingen wilt veranderen, moet je ze een soort van keuze geven, en na die keuze wordt het hun verandering in plaats van dat je het oplegd.

\textbf{Bojhan:} Je kan mensen nooit iets opleggen, het zijn vrijwilligers, dus als ze het niet leuk of goed vinden, dan doen ze het gewoon niet. Het is inderdaad vooral overtuigen. Het is heel moeilijk, want mensen die werken er dagelijk mee, en niet alleen aan. Ze hebebn dus al heel veel patronen die in hun hoofd zitten en die ze toepassen, zelfs al zijn die compleet fout.

\textbf{Kilian:} Je hebt vaak dat je weet dat iets niet optimaal is, maar dat gebruikers het al gewend zijn en het snappen. Je zit dan met de vraag of je het aan moet passen of niet.

\textbf{Bojhan:} Ja, dat is moeilijk. Op een live website kan je altijd wel kleine aanpassingen doorvoegen. Je kan dan ook geen grote aanpassingen doen, want dan moet iedereen opnieuw getraint worden. Wat ik zelf wel mooi vind is de ebay methode. Die hadden een formulier, en ze merkte dat dat niet goed is. Maar de powerusers waren daar heel erg aan gewend. Ze hadden toen een knop aan de rechterkant toegevoegd met de tekst `show me the new form'. Ze zagen toen dat langzaam werdt het zestig procent, zeventig procent gebruik, en toen zijn ze over gegaan.

In drupals geval is dat natuurlijk niet mogelijk, omdat we echt een applicatie zijn. Drupal maakt wel grote releases, niet als wordpress, die doet iedere keer kleine versies die een beetje veranderen, en wij doen grote versies die een hoop veranderen. Dus we hebben wel de instelling `we weten dat het beter is, dus leer het maar overnieuw'. Het zou heel mooi zijn als we change management konden doen, maar het kan niet.

\textbf{Kilian:} Zie jij recente ontwikkelingen op het gebied van usability die je goed of slecht vind?

\textbf{Bojhan:} Wat ik zelf zie is de trend dat vroeger er vooral gericht werd op de beginnende gebruiker, en het shift nu ietsje meer naar de powerusers. Dat zie je bij flickr, bij delicious, bij twitter, overal. Er komen gewoon steeds meer functionaliteiten die enkel de poweruser bedienen. Qua usability is dat wel een grote verandering, want powerusers zijn gewoon een ander soort gebruikers.

\textbf{Kilian:} Waarom denk je dat er meer op de powerusers wordt gericht?

\textbf{Bojhan:} Ik denk vooral omdat die community maken. Vooral bij sites als Flickr is het belangrijk. Dat je een groot aantal powerusers hebt die de content maken om voor de middengroep te bekijken. Ik denk dat ze daar tot de realisatie zijn gekomen dat dat echt belangrijk is. Vroeger had je community sites die ten onder gingen omdat er beslissingen werden genomen die totaal tegen de powerusers in waren. Ik denk ook dat meer mensen powerusers worden.

\textbf{Kilian:} Als je kijkt naar verschillende onderzoeksmethodieken, zijn er dan die je heel handig vind, en verschilt dat met het budget wat je hebt?

\textbf{Bojhan:}  Wat ik veel toepas zijn expert reviews. Heel veel budget kan er in die initiele gebruikerstests uitgaan, maar je je kan ook vaak al heel veel doen zonder die test. Een goede expert review staat vaak aan de basis van wat ik doe. Bij mij komt usability testen pas echt in beeld wanneer er iets gemaakt gaat worden.

Wat ik zelf heel veel doe is het `concepts en relationships' idee. Ik kijk dan welke concepten er in een applicatie leven, en dan bekijk ik de relaties tussen die twee. Hoe liggen de workflows? en hoe kunnen we een interface maken waarbij zowel de workflow als de efficiency verbeterd worden? Daarvoor moet je vaak een heel goed begrip hebben van het concept. Dat is meer een informatie modelling methode.

\textbf{Kilian:} Hoe doe je dat precies?

\textbf{Bojhan:} Een heel basaal voorbeeld: in een logistiek systeem kijk je heel basaal naar het concept van uitgaande goederen. Dan kijk je naar resources en de planning en alle aparte delen die eigelijk altijd bij elkaar zitten maar waar ander soort mensen op zitten. Het is vooral gericht op de verschillende rollen in zo'n systeem, en die komen meestal niet samen. Er komen dan ook heel veel communicatieproblemen naar voren, en dan moet je echt kijken naar de mensen die met de concepten werken, en hoe de relaties tussen die mensen nu ligt, en hoe wordt dat gemodelleerd in een applicatie. Meestal is dat heel slecht. Dat idee is vooral bij organisaties heel gemakkelijk over te brengen. Ze snappen het dan sneller.

Meestal kom je er om iets mogelijk te verbeteren, ze gaan er van uit dat het eigelijk al heel goed is, en meestal is dat niet het geval. Voor mij is het belangrijk dat je met usability toch de business value kan bewijzen.

\textbf{Kilian:} Hoe zie jij de toekomst van usability onderzoek en user experience?

\textbf{Bojhan:} Ik denk dat het meer van het web afgaat en integreerd met onze omgeving. Ik zie dat interaction design meer de shift maken naar je omgeving en de tools die je met je meedraagt, en niet enkel achter je pc. Ik denk dat dat een goede beweging is, vooral omdat het persoonlijk contact dan toch weer belangrijker wordt.

\textbf{Kilian:} Hoe zie jij dat qua ontwerp van websites? Wordt de data belangrijker en wordt die door devices on een bepaalde layout gezet?

\textbf{Bojhan:} Niet per se een open data model, maar meer het idee dat jij niet het perfecte visualisatie van de data kan hebben. De overheid van de VS doet dat nu heel goed, die pusht de data nu uit zodat anderen daar wat mee kunnen. Het worden veel meer centralised points van data van centralised points van access. Bij twitter gebeurd dat al een beetje. De site is niet meer de main hub, het is meer de verzamelplaats, en jouw applicatie op je telefoon is de acces point. Dan kan het ook gemakkelijker die omslag maken naar jouw omgeving, omdat het enkel die data hoeft te hebben. Maar daarvoor moet er nog wel een aardige verandering zijn in hoe de networking sites met elkaar werken.

