\chapter{Lab tests analyses}
    \label{labtestsappendix}

\section{Claudia Engelsman}
\textbf{19 jaar, past bij persona van Mariette.}

\subsection{Opdracht 1}
  Ziet het zoekveld niet en scrollt initieel niet ver genoeg terug naar boven om het te zien. Klikt een aantal keer direct naast het zoekveld op andere opties en komt uiteindelijk op een toplijst, waaruit ze een programma kiest. Eenmaal op de pagina is het schrijven van een review geen probleem

\subsection{Opdracht 2}
  Zoekt naar `websites' in het zoekveld, maar de site reageert erg traag, zo erg dat ze zich afvraagt op het werkt. eenmaal op de zoekpagina klikt ze vrij snel op adobe dreamweaver, totdat ze zich realiseert dat dat geen website is. Ze gaat weg van de zoekpagina en is verward over de verdeling tussen websites en programma's. Via het eigen profiel klikt ze op twitter om dit vervolgens in haar browser toe te voegen als favorite. Hierop heb ik de benaming aangepast

\subsection{Opdracht 3}
  Ze klikt doelgericht naar het eigen profiel en bekijk de pagina. Vervolgens klikt ze bij de grafiek op "today".

\subsection{Opdracht 4}
 Ze klikt direct op people maar kan het daar niet vinden, ze klikt op het dropdown menu van you en op "find en invite people". Ze staat op het punt om het op te geven en klikt door naar het dashboard. Ze zegt dat er wel software recommendations zijn, maar niet gebruikers. Ze scrollt verder naar beneden en meld "oh, ik wist niet dat je ook buren kon hebben", maar ziet dit niet als zijnde `mensen die op je lijken', dit merkt ze ook op. Ze klikt door naar software recommendations, waar "neighbours" wordt uitgelegd. Op het profiel heeft ze geen probleem de persoon als contact toe te voegen.

\subsection{Opdracht 5}
 Ze zoekt teams bij you en bij dashboard, vervolgens bij software en dan naar categorien, die ze als teams ziet. ze klikt door naar een programme en merkt dan op dat het `zeker geen team is'. Ze zoekt in de zoekbalk naar `team', kijkt moeilijk en klikt dan op de teams lijst. Ze zoekt naar twee applicaties, rollercoaster tycoon en spore, waarbij de laatste een team heeft. Eenmaal op de teampagina klikt ze vrij snel op `join this team.'

 \subsection{Algemeen}
  Over het algemeen liggen voor Claudia de meeste problemen in de terminologie en de navigatie. Wanneer ze eenmaal op een pagina zit, vind ze vaak snel waar ze naar zoekt.

\section{Mark van der Ham}
\textbf{20 jaar, past bij persona van Tom.}

\subsection{Opdracht 1}
Klik gelijk door naar Reviews, en zoekt daarna naar Itunes, scrolled vrij lang door de lijst. In de nabespreking gaf hij aan dat het te druk was en dat de verschillende programma's die op itunes leken hem verwarden. Na een aantal keer scrollen klikt hij op de eerste hit. Zodra hij op de itunes pagina zit vind hij het reviewformulier snel

\subsection{Opdracht 2}
Zoek naar een site die niet op wakoopa staat, met de volledige url. Na een opmerking over wat voor sites het moet gaan, zoekt hij op hyves en vind snel de favorite-knop. Vult een erg lange tekst in ondanks het kleine invoerveld.

\subsection{Opdracht 3}
Klik vrij zeker op you, scrollt naar Usage, klikt op Today, maar denkt daarna dat het 'usage in the last hour' het gebruik vandaag is en laat het verder daarbij.

\subsection{Opdracht 4}
Bekijkt vrij lang de persoonlijke profielpagina. Zoekt dan op "developer" en bekijkt het profiel van een van de hits. Gaat dan terug naar het eigen profiel, naar usage and contacts. Zegt dat hij iets van aanbevelingen zoekt maar dat niet kan vinden. Klikt door naar find and invite en ziet dan een link naar recommendations. Leest het tekstje en zegt dat hij het gevonden heeft.

Op het profiel zegt hij dat het al een contact is en wijs op een lege plek naar de block button. Scrollt omlaag en omhoog en ziet dan de "add link".

\subsection{Opdracht 5}
klikt door naar teams, en zoekt naar ultrabeat, waar geen team voor wordt gevonden. Zoekt vervolgens naar 'call of duty' via het zoekveld, die automatisch op software in plaats van developers zoekt. Klikt door naar software, en doet dit een aantal keer. Klik door naar users en merkt op dat het geen team is. Klikt terug naar de teams pagina en klikt op gamers. Op de teampagina klikt hij direct op "join this team".

\subsection{Algemeen}
Mark heeft de dashboard, je eigen startpagina, helemaal niet gezien. Dit had wellicht geholpen bij het vinden van in ieder geval de aanbevelingen. Een grote fout was dat bij het zoeken niet onthouden werd naar wat voor soort dingen gezocht werd, en dat dit er niet duidelijk bovenstond. In opdracht 5 lukte het Mark daarom niet goed een team te vinden.

\section{Mark Dekkers Los}
\textbf{22 jaar, past bij persona van Andreas.}

\subsection{Opdracht 1}
Klik op software en gaat dan naar categories. Vind daar de zoekbalk. Zoekt op quicktime en klikt op het eerste resultaat. Klikt daar gelijk door naar Reviews, en dan in de rechterbovenhoek op "Write one!". Schrijft een review, kijkt even naar de andere reviews en voegt dan ook een rating toe.

\subsection{Opdracht 2}
Klikt op het logo om naar het dashboard, klikt de verschillende menu's aan, en zoekt vervolgens op "website". Zoekt door de resultaten en vindt Facebook. Scrollt over de hele pagina en weer terug, en klikt dan op favorite.

\subsection{Opdracht 3}
Klikt op het profiel en kijkt naar recently used, type of usage and ziet dan "today" bij de grafiek. Klikt op today.

\subsection{Opdracht 4}
Klikt op people, kijkt daar even en wilt gaan zoeken, maar kan niet op een term geven. Vraagt op welke manier ze op elkaar moeten lijken, met het antwoord dat Wakoopa dat aangeeft. Klikt bij people op reviews, en twijfelt vervolgens in het dropdown menu tussen find and invite en recommendations. Klikt op recommendations, vind de neighbours en klikt direct op het toevoegen icoontje.

\subsection{Opdracht 5}
Gaat naar teams, zoekt op "call of duty", merkt op dat het team spaans is en klikt vrij direct op join this team.

\subsection{Algemeen}
Er zijn geen algemeen opvallende dingen aan deze labtest.

\section{Jorn van Schaik}
\textbf{20 jaar, past bij persona van Tom.}

\subsection{Opdracht 1}
Bekijkt het dashboard en klikt op verschillende plekken. Gaat naar het zoekveld en typt dan een zin in, maar zoekt niet. Klikt vervolgens op quicktime player. Geeft later aan dat hij het logischer vondt als je eerst aangaf een review te willen schrijven, en daarna pas waarover in plaats van andersom. Op de softwarepagina vind hij erg snel de review-mogelijkheid.

\subsection{Opdracht 2}
gaat naar het zoekveld en zoekt op een volledige site (url) en vind dan niks. Gaat terug naar de dashboard en klikt dan op "Find and invite". Bekijkt de dropdowns en kijkt verder rond. Vraagt of het op Wakoopa moet. Klikt vervolgens zonder iets te typen op de zoekknop, waardoor er naar "search for..." wordt gezocht. Klikt dan op Google search en vind op de pagina heel snel de favorite knop

\subsection{Opdracht 3}
Klikt direct op de dropdown bij `you' en klikt dan op settings en gaat naar favorites. Bekijkt nogmaals de dropdowns, scrollt naar beneden en klikt op help. Scrollt heen en weer en klikt uiteindelijk op You. Ziet de grafiek, maar selecteert de `top used applications' in plaats van category usage.

\subsection{Opdracht 4}
Klikt direct in de dropdown bij `you' op recommendations, leest het stuk over neighbours, klikt door naar de gebruiker en klikt daar snel op `add as contact'

\subsection{Opdracht 5}
Bekijkt alle dropdowns nogmaals en klikt dan door op `teams'. Bekijkt de pagina maar zoekt niet. Klikt uiteindelijk op een groep en klikt vervolgens direct op `join this team'

\subsection{Algemeen}
Jorn klikte veel en vond daardoor veel functies al voordat hij ze nodig had. Hierdoor kreeg hij snel feeling voor hoe de site in elkaar zat. Opvallend is dat, hoewel hij de zoekfunctie wel heeft gebruikt, en iets in het zoekveld getypt, heeft hij niet bewust naar iets gezocht. Na afloop gaf hij aan dat hij dit gemakkelijker vond.

